
\subsection{Concetti fondamentali sulle fonti giuridiche}

Il diritto è un sistema di regole che esiste per disciplinare il vivere sociale e risolvere eventuali conflitti tra gli uomini. 
Regole che aiutano non tanto a determinare chi ha ragione e chi ha torto ma piuttosto a soddisfare gli interessi di diverse parti in modo tale da determinare che interesse prevale e che interesse soccombe (es. interesse dello sviluppatore di un software ad essere pagato prevale sull'interesse dell'utente a scaricarlo gratuitamente).

A livello atomico una norma giuridica è un'unità elementare del sistema del diritto. 

Il diritto si divide in:
\begin{itemize}
    \item civile
    \item penale
    \item amministrativo
\end{itemize}

La disciplina del diritto delle nuove tecnologie è trasversale a questi ambiti.


La norma è un comando/precetto generale (non riferito a specifici individui) e astratto (non riferito a casistiche specifiche) che impone o proibisce un certo comportamento e stabilisce certi diritti/obblighi/oneri. Le leggi sono divise in articoli, a loro volta suddivisibili in commi, ognuno dei quali può contenere una o più norme.

Es. legge di protezione del diritto d'autore e di altri diritti connessi al suo esercizio - legge del 22 aprile 1941, n.633. il secondo comma di questa legge è stato introdotto nel 1992; la datazione originale della legge permane nonostante le modifiche successive.

ATTENZIONE: nonostante le leggi possano apparire descrittive, esse svolgono in realtà una funzione precettiva, stabilendo diritti e doveri. 


Sistema delle fonti del diritto:
\begin{enumerate}
    \item Trattato della comunità europea e regolamenti comunitari
    \item Costituzione e leggi costituzionali
    \item Leggi ordinarie dello Stato e atti aventi forza di legge 
    \item Leggi regionali
    \item Fonti secondarie e regolamenti
    \item Usi e consuetudini
\end{enumerate}

Le fonti di grado superiore non possono essere contraddette da fonti di grado inferiore. 

A livello europeo possono nascere possibilità di conflitti: anche a livello europeo esistono trattati costitutivi affiancati a regolamenti comunitari. L'UE deve essere organizzata in modo tale che i trattati non interferiscano con le costituzioni dei vari Stati e gli atti normativi UE non solo non possono contraddire l'atto costitutivo UE ma non possono nemmeno contraddire le normative nazionali. In alcuni ambiti ciò è facile, mentre in altri ambiti è molto complesso legiferare a livello europeo e avere una garanzia di legittimità a livello nazionale esteso a tutte le nazioni membri dell'UE. 
Per questo si distingue tra:
\begin{itemize}
    \item Regolamenti: norme di legge che stanno a grado di priorità più alto ma riguardano aspetti operativi che difficilmente confliggono con aspetti costituzionali dell'UE e degli Stati membri
    \item Direttive: linee guida o meta-leggi a cui devono sottostare le leggi che verranno emanate su quei temi in ogni Paese
\end{itemize}

Un regolamento è valido su tutto il territorio europeo, poiché proviene da un consenso da parte di tutti gli Stati membri. Le direttive, al contrario, richiedono il lavoro di ciascuno Stato: per questo non si tratta di leggi che entrano in vigore così come sono ma obbligano gli Stati a legiferare sui temi su cui sono state emanate rispettando le direttive stesse.

\subsubsection{Atti aventi forza di legge}

\begin{itemize}
    \item Decreto legge: emanato dal governo in casi di necessità straordinaria o urgenza; atti aventi forza di legge ma decadono se non vengono convertiti in legge dal Parlamento entro 60 giorni
    \item Decreto legislativo (leggi delegate): emanato dal governo ma sono direttamente delle leggi. L'esecutivo ha una quota parte di potere legislativo o per situazioni emergenziali oppure per delega del Parlamento
\end{itemize}

\subsubsection{Regolamenti}
Emanati dal governo o da altre autorità (regioni, province, comuni, banca d'Italia, AGCOM, …)
\begin{itemize}
    \item Di esecuzione: regolamento in particolari materie già disciplinate dalla legge
    \item Indipendenti: regolano materie non ancora regolate da alcuna legge
\end{itemize}

N.B. mai un regolamento può essere contrario a una legge

\subsubsection{Efficacia della legge nel tempo}
Leggi e regolamenti entrano in vigore con la pubblicazione in Gazzetta Ufficiale (15 gg dopo la data di pubblicazione a meno che non sia specificato altrimenti). Alcune leggi (eccetto quelle penali) possono essere emanate con valore retroattivo. 

La legge cessa di avere efficacia per:
\begin{enumerate}
    \item Abrogazione espressa: espressa disposizione di legge successiva, referendum popolare o sentenza di illegittimità costituzionale
    \item Abrogazione tacita: incompatibilità con una nuova disposizione di legge oppure nuova legge che regola l'intera materia includendo gli ambiti di cui si occupava la legge precedente
\end{enumerate}

\subsubsection{Efficacia della legge nello spazio}
Efficacia nazionale della legge (statualità del diritto).

Possibili conflitti tra norme di Stati diversi. Soluzioni:
\begin{itemize}
    \item Diritto internazionale
    \item Accordi
    \item Convenzioni
\end{itemize}

Normalmente si applica la legge nazionale ai soggetti ma legge del luogo agli oggetti (problematico nel caso digitale: dati di utenti italiani su server all'estero sottostanno al diritto italiano o locale?).

\subsubsection{Interpretazione della legge}
Interpretazione può essere:
\begin{itemize}
    \item Letterale: si deduce l'applicazione della legge dal contenuto delle parole
    \item Teleologica: da un testo di legge estrapola l'intenzione del legislatore. A partire da un testo di legge si può andare verso un significato più ampio (interpretazione estensiva) di quello letterale o più ristretto (interpretazione intensiva)
    \item Analogia: se manca una regola specifica per risolvere una controversia si fa riferimento alle disposizioni che regolano casi simili o materie analoghe
\end{itemize}

È possibile che vi siano lacune nella legge dovute alla rapida evoluzione dell'oggetto della legge, come si vede nell'ambito della rapida evoluzione della tecnologia, con cui spesso l'evoluzione della legge non riesce a stare al passo.


\subsubsection{Sentenza e gradi di giudizio}
Mentre la legge è generale ed emanata da un organo legislativo, la sentenza è emanata da un giudice ed è valida solo per uno specifico caso, vincolando solo le parti in causa. Si tratta di un provvedimento concreto che complementa la natura astratta della legge.

I gradi di giudizio in Italia sono tre:
\begin{itemize}
    \item Giudice di primo grado
    \item Giudice di secondo grado: riesame della sentenza di primo grado (corte d'Appello)
    \item Giudice di terzo grado: non entra nel merito della specifica controversia ma effettua un controllo sulla legittimità delle indagini/sentenze dei precedenti gradi di giudizio (corte di Cassazione)
\end{itemize}

La sentenza non costituisce un precedente giudiziario vincolante. Nel diritto anglosassone è previsto che il giudice con la propria sentenza ``completi" il diritto rendendo vincolante la propria sentenza al fine di sentenze future.
