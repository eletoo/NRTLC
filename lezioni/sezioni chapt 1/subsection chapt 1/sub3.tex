
\subsection{Organi dell'Unione Europea}
\begin{itemize}
    \item Parlamento: rappresenta tutti i cittadini UE; i membri sono eletti ogni cinque anni; ha potere di iniziativa legislativa (solo sulle materie stabilite dai trattati) ma principalmente ha potere legislativo insieme al Consiglio dell'UE.
    \item Commissione: 27 commissari, assimilabili a ministri, uno per ogni Stato membro. Essi lavorano per riflettere su cosa sarebbe meglio per l'UE nel suo insieme. La commissione rappresenta e tutela gli interessi dell'UE indipendentemente dalle direzioni prese dalle politiche interne a ogni Stato membro. Il compito è elaborare proposte per nuove leggi europee da sottoporre al Parlamento europeo e al Consiglio (iniziativa legislativa). Attuale presidente: Ursula Von Der Leyen
    \item Consiglio: 
    \begin{itemize}
        \item consiglio dell'UE: riunisce i ministri dei vari Stati; potere legislativo attuato in co-decisione con il Parlamento europeo
        \item consiglio Europeo: consiglio che riunisce i leader dei Paesi europei, non ha funzione legislativa ma si tratta di un organo di strategia 
        \item consiglio d'Europa (non organo dell'UE). 
    \end{itemize}
    \item Corte di giustizia: garantisce che tutti i paesi dell'unione rispettino gli atti legislativi e che tali atti europei rispettino le norme dei vari stati
\end{itemize}