\section{Il Terzo pacchetto quadro del 2009}

\subsection{Contesto}
Questo pacchetto nasce in un contesto in cui il quadro internazionale era eterogeneo e complesso; infatti l’applicazione delle direttive del 2002 è stata lunga e piena di ostacoli, anche perché queste non avevano un sufficiente grado di dettaglio, inoltre nel 2004 l’UE si era allargata, con un conseguente peggioramento dell’eterogeneità normativa. Si decide quindi di definire un terzo pacchetto di normative che risulta più soft dal punto di vista degli obblighi e orientato all’autoregolamentazione del mercato concorrenziale.

\subsection{Impianto}
Il nuovo pacchetto consta quindi di:
\begin{itemize}
    \item Dir. 2009/140/CE, che va a modificare le direttive del 2002 riguardanti il quadro normativo, l’accesso    alle reti e le autorizzazioni.
    \item Dir. 2009/136/CE che va a modificare le precedenti direttive relative al S.U. e alla vita privata.
    \item Un regolamento che istituisce l'Organismo dei regolatori europei delle comunicazioni elettroniche (BEREC), organismo in realtà molto depotenziato rispetto all’idea originale (in pratica non si è riusciti a    replicare a livello europeo il ruolo delle autorità nazionali).
    \item Una raccomandazione relativa all’accesso alle reti di nuova generazione (NGA).
\end{itemize}
 
I problemi dell’esperienza della riforma del 2002 hanno portato ad un parziale ritorno alle origini in termini di modello con cui approcciare la liberalizzazione, comportando il rischio della vanificazione degli obiettivi della concorrenza sui mercati a scapito soprattutto del consumatore; nasce l’esigenza di una fase di armonizzazione con un intervento comunitario di grande impatto: si pensa ad un accentramento di poteri nelle mani dell’autorità europea di settore, ma questo è ostacolato dagli stati membri, quindi si tengono infine validi gli obiettivi iniziali della terza riforma, ma ne vengono depotenziati gli strumenti.


\subsection{Obiettivi}

Gli obiettivi del terzo pacchetto sono:
\begin{itemize}
    \item Armonizzazione del mercato interno: Continua l’assenza di un mercato unico delle comunicazioni     elettroniche, si vuole quindi una normativa più dettagliata che lasci meno discrezionalità agli stati    membri, pensando all’istituzione di un ente europeo per il coordinamento delle autorità nazionali.
    \item Creazione dell’organismo dei regolatori: La creazione di questo organismo (ETMA in origine) è    stata ritardata di due anni, e si è “tramutata” nel BEREC con un forte depotenziamento rispetto all’idea di partenza.
    \item Gestione dello spettro radio (digital dividend): Lo spettro è considerabile come una risorsa limitata ma importante: diventa così necessario gestirlo in modo efficiente ed efficace; di grande interesse è il cosiddetto dividendo digitale, ovvero la maggiore disponibilità di spettro dovuta all’uso più efficiente delle risorse (passaggio dalla TV analogica a quella digitale); importante il principio di neutralità che impone procedure aperte e non discriminatorie per l’uso delle frequenze rese disponibili dall’evoluzione tecnologica (eliminazione del legame mezzo-servizio).
    \item Servizio universale (digital divide): Si tratta di una materia controversa e articolata e che non è stata armonizzata né approfondita dal legislatore comunitario; si forniscono alcuni chiarimenti su temi specifici, tra cui la classificazione della connessione alla rete pubblica a prezzo accessibile come S.U. (si tratta del problema di digital divide, ovvero del divario che c’è tra chi ha accesso a tecnologie dell’informazione e chi no), il fatto che gli obblighi del S.U. possano essere adempiuti da più di un operatore e il fatto che possono proporsi per fornire il S.U. operatori che usano reti fisse o mobili senza distinzioni.
    \item Revisione delle regole dei mercati e degli obblighi degli operatori: Vengono imposte scadenze periodiche per l’esame dei mercati da parte delle Autorità nazionali, con possibilità di suddivisione sub-nazionale dei mercati e di disposizione di strumenti di controllo più raffinati.
    \item Separazione funzionale: L’operatore è verticalmente integrato (cioè è presente dall’ingrosso al dettaglio su vari servizi) ed è tenuto a collocare le attività di fornitura all’ingrosso di prodotti di accesso in un’entità commerciale indipendente sul piano operativo, in modo da fornire prodotti e servizi a tutte le imprese agli stessi termini e condizioni; la separazione può essere frutto di una decisione autonoma (e quindi l’Autorità nazionale è tenuta a valutarne le conseguenze sui mercati) oppure imposta dall’Autorità (che si giustifica come misura contro la discriminazione in alcuni mercati).
    \item Separazione funzionale: L’operatore è verticalmente integrato (cioè è presente dall’ingrosso al dettaglio su vari servizi) ed è tenuto a collocare le attività di fornitura all’ingrosso di prodotti di accesso in un’entità commerciale indipendente sul piano operativo, in modo da fornire prodotti e servizi a tutte le imprese agli stessi termini e condizioni; la separazione può essere frutto di una decisione autonoma (e quindi l’Autorità nazionale è tenuta a valutarne le conseguenze sui mercati) oppure imposta dall’Autorità (che si giustifica come misura contro la discriminazione in alcuni mercati).
    \item Sicurezza nelle reti: Tema centrale ed innovativo in partenza, ha poi perso rilievo.
    \item Reti di nuova generazione (NGN): Tema centrale e molto dibattuto tra chi sosteneva che servissero norme per l’accesso e chi sosteneva invece che queste fungessero da deterrente all’innovazione; l’UE risponde con un compromesso.
\end{itemize}

\subsection{Strumenti}
Anche gli strumenti del terzo pacchetto sono stati modificati rispetto all'idea originaria:
\begin{itemize}
    \item Gli strumenti normativi hanno visto un depotenziamento della normazione ex-ante e l'istituzione di un organismo europeo dei regolatori.
    \item Gli strumenti esecutivi hanno visto un’imposizione di una nuova disciplina sanzionatoria che consente all’Autorità di comminare sanzioni per far valere il rispetto delle condizioni delle autorizzazioni generali o dei diritti d’uso.    
\end{itemize}

\subsection{Contenuti}
Per quanto riguarda i contenuti del terzo pacchetto:\newline

Per le NGN si propone una soluzione di compromesso tra accesso ad un’unica infrastruttura e concorrenza infrastrutturale tra reti; in questo contesto risultano significativi gli orientamenti del terzo pacchetto in merito alla neutralità tecnologica, apertura ad una separazione funzionale negoziata anziché imposta, flessibilità e adattabilità, delega agli stati per la definizione della velocità minima di collegamento a Internet e limite del digital divide.\newline

Si predispongono le tappe verso un rinnovato mercato mobile; in particolare le elevate tariffe del roaming e di terminazione erano inizialmente giustificate come necessarie a finanziare gli investimenti in nuove tecnologie, ora però si pensa ad intervenire per ridurre tali tariffe, piuttosto si deve cercare il finanziamento nel complesso mondo dei servizi a valore aggiunto (fuori da quelli base).\newline

Per risolvere le controversie tra operatori la terza riforma conferisce poteri di arbitrato obbligatorio all’Autorità nazionale, ma che in Italia risultano in illegittimità/anticostituzionalità con un conseguente depotenziamento.\newline

Il consumatore viene messo al centro come parte delle nuove riforme; in particolare vengono curati la privacy e gli strumenti di conciliazione. Ora l’accesso a Internet diviene libertà fondamentale e nasce la tutela contrattuale, con un maggiore dettaglio per la trasparenza delle informazioni e la qualità dei servizi e con una maggiore efficacia nel controllo di pubblicità ingannevoli e pratiche commerciali scorrette.