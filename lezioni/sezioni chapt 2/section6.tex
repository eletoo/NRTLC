\section{Recepimento della direttiva 2009/140/CE e modifiche al CCE}

L’Italia recepisce in ritardo il terzo pacchetto comunitario, con il decreto legislativo n° 70 del 28 maggio
2012, che modifica in modo sostanziale il CCE; in particolare mira a:
\begin{itemize}
    \item Promuovere investimenti efficienti e innovazione delle infrastrutture → Viene introdotta la possibilità per le autorità nazionali di imporre la condivisione di infrastrutture tra operatori.
    \item Promuovere una gestione efficiente e flessibile e coordinata dello spettro radio → Viene introdotto il principio di neutralità tecnologia e dei servizi (non discriminazione tra tecnologie).
    \item Rafforzare i diritti degli utenti in materia di trasparenza nei rapporti con i fornitori di servizi → Rafforzamento della portabilità del numero e maggiori obblighi di trasparenza e qualità del servizio
    \item Rafforzare le prescrizioni in tema di sicurezza e riservatezza delle comunicazioni → In particolare si vuole rafforzare la protezione dei dati personali.
\end{itemize}

Rimangono le stesse criticità nello scostamento dalla legislazione europea già evidenziate nel 2003: separazione tra TLC e radiotelevisione, mantenimento del differimento del nuovo regime autorizzatorio e mantenimento di poteri rilevanti al ministero competente. Queste vengono arricchite da un forte attrito tra direttiva comunitaria e disciplina nazionale riguardante il doppio livello di pianificazione delle frequenze (piano di ripartizione e piano di assegnazione) ed il suo mantenimento; questo ostacola il recepimento del principio di neutralità delle frequenze, limitandone l’applicabilità alla sola banda assegnata ai servizi di
comunicazione elettronica e di conseguenza il processo di convergenza perseguito dall’UE.