\section{Attualità delle reti a banda larga nel mercato unico europeo}

Recentemente in Europa hanno definito degli obiettivi di lungo termine volti a sviluppare il mercato della
banda larga; in particolare:
\begin{itemize}
    \item Agenda digitale Europea (2010) per il 2020 → Tra gli obiettivi fissati si impone la promozione dell’accesso a Internet veloce e superveloce a tutti i cittadini europei e a prezzi competitivi; a tal fine si dovranno creare reti NGA finanziate da fondi europei. Inoltre a tutti gli Stati membri è stato imposto entro il 2015 il raggiungimento della copertura di alcuni servizi digitali (e-commerce, e-government, \dots) ed entro il 2020 di:
    \begin{itemize}
        \item Garantire la copertura della banda larga veloce ($>$30 Mbps) per tutti i cittadini.
        \item Garantire la copertura della banda larga ultraveloce ($>$100 Mbps) per almeno il 50\% dei cittadini.
    \end{itemize}
    Nel 2016 la Commissione Europea ha condotto un’analisi dei progressi degli Stati membri (sulla base di connettività, capitale umano, uso di Internet, integrazione delle tecnologie digitali e servizi pubblici digitali) che ha confermato che l’Italia si trova tra i paesi con le più basse prestazioni digitali (25° posizione); questo problema è dovuto non solo all’infrastruttura ma anche ai cittadini che “si accontentano” e non incentivano gli operatori ad investire in servizi migliori.
    \item Agenda digitale Italiana (2012) → Si definisce la Strategia per la banda Ultralarga che prevede interventi finanziati da risorse pubbliche e differenziati per ogni tipologia di area:
    \begin{itemize}
        \item Cluster A, corrisponde alle aree nere NGA, dove si ha più di un operatore interessato ad investire.
        \item Cluster B, corrisponde alle aree grigie NGA, aree con un unico operatore interessato ad investire.
        \item Cluster C, corrisponde alle aree bianche NGA, cioè le aree a fallimento di mercato, con nessun operatore interessato ad investire, ma con una infrastruttura per la banda larga con capacità inferiore a 30 Mbps.
        \item Cluster D, corrisponde alle aree bianche NGA, cioè le aree a fallimento di mercato, con nessun operatore interessato ad investire e in cui solo l’intervento pubblico diretto può garantire alla popolazione residente un servizio con capacità superiore a 30 Mbps.
    \end{itemize}
    Successivamente si decide di adottare il Piano di investimenti per la diffusione della banda Ultralarga, un piano che coinvolge l’intero territorio italiano e volto a impiegare i fondi pubblici compatibilmente con le aree sopra definite. A tal fine la società del Ministero dello sviluppo economico detta Infratel Italia S.p.A., indice delle gare per la realizzazione e la gestione di infrastrutture adeguate (2016), al fine di ridurre il digital divide nelle aree a fallimento di mercato.

    Crea l’Agenzia Italiana per il Digitale (Agid), che dovrebbe garantire il raggiungimento degli obiettivi;tuttavia ha fatto ben poco, nel tempo si è rivolta a diversi enti esterni per consulenza, ma nel frattempo non ha fatto nulla di pratico; nel 2013 è stata denunciata per danno erariale. Risulta uno dei motivi principali per cui l’Italia è così in ritardo rispetto al resto dell’Europa.
\end{itemize}

\subsection{Digital Single Market}
Nel 2013 viene pubblicata la Proposta di Regolamento per la riduzione dei costi di realizzazione della banda larga, successivamente convertita in Decreto del fare (d.l. 69/2013). Quest’ultimo contiene importanti novità, in particolare viene abolito il Patentino degli installatori e introdotta la liberalizzazione del Wi-Fi (accolto molto positivamente soprattutto dai piccoli operatori).\bigskip

Viene inoltre proposto un regolamento da parte della Commissione europea contenente un pacchetto legislativo di riforma del settore TLC (“Continente connesso: costruire un mercato unico delle TLC”): si tratta di un insieme di modifiche che si basano su due principi cardine, ovvero la libertà di fornire consumare servizi digitali ovunque ci si trovi all’interno dell’UE e la non discriminazione.\bigskip

Successivamente viene emanata la Dir. 2014/61/UE che introduce misure volte a ridurre i costi dell’installazione di reti ad alta velocità permettendo l’uso di infrastrutture fisiche esistenti e metodologie più efficienti di integrazione di quelle nuove.\bigskip

Viene infine definita la Strategia per il mercato unico digitale in Europa (COM2015) che prevede uno stanziamento di opportuni fondi annuali per la realizzazione di profonde riforme volte a migliorare l’accesso dei consumatori, a creare un contesto favorevole e paritario e a massimizzare il potenziale di crescita
dell’economia digitale.\bigskip 

Sia per la direttiva del 2014 che per il piano strategico, l’Italia si ritrova tra i primi ad attivarsi per ricevere le direttive in modo da accelerare il recupero del divario rispetto agli altri Stati membri.
