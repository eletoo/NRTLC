\subsection{Apertura del mercato a più operatori}

\subsubsection{La nascita di Telecom Italia}
Primo provvedimento: legge 58/1992 (Disposizioni per la riforma del settore delle Telecomunicazioni). 

Soppressione dell'Azienda di stato per i servizi telefonici e contestuale Separazione delle funzioni di controllo (Ispettorato generale delle TLC) e di gestione (Iritel) del servizio telefonico e creazione di un gestore unico dei servizi dei TLC (Telecom Italia).


Telecom Italia nasce il 27 luglio 1994 con funzioni di gestore unico delle telecomunicazioni dalla fusione di cinque società operanti nel settore telefonico (SIP, Iritel, Italcable, Telespazio e SIRM) appartenenti al gruppo IRI-STET. Rimanevano poteri ministeriali su funzioni molto importanti quali:
\begin{itemize}
    \item Assegnazione delle frequenze
    \item Rilascio delle concessioni
    \item Fissazione delle tariffe
    \item \dots
\end{itemize}

La privatizzazione di Telecom Italia si avvia nel '97.

Passaggi decisivi per il completamento dell'adeguamento alle direttive europee:
\begin{itemize}
    \item Open Network Provision: operatori non proprietari della rete possono fornire servizi sfruttando la rete di terzi 
    \item Liberalizzazione dei servizi
    \item Legge 248/97: istituzione dell'autorità per le garanzie nelle comunicazioni e norme sui sistemi di telecomunicazioni e radiotelevisivo
    \item D.p.r. 318/97: regolamento di attuazione delle direttive comunitarie nel settore delle telecomunicazioni
\end{itemize}

In sintesi, viene aperto il mercato a più operatori, non vengono più poste delle concessioni ma autorizzazioni generali e licenze individuali, si affermano principi di trasparenza gestionale (es. trasparenza del bilancio aziendale), istituzione di un'apposita Autorità di garanzia e definizione di nuove regole antitrust.

\subsubsection{Superamento del regime concessorio}
Abbandono fin dove si può dell'istituto della concessione, che porta all'abbandono di diritti esclusivi e speciali. 
Nel d.p.r. 318/97 si specifica la distinzione tra la concessione e l'autorizzazione generale:
\begin{itemize}
    \item Autorizzazione generale: autorizzazione che consente l'uso dei diritti da essa derivanti indipendentemente da un'autorizzazione esplicita da parte della pubblica autorità
    \item Licenza individuale: autorizzazione che deve essere esplicitamente rilasciata dall'autorità e che conferisce ad un'impresa diritti specifici
\end{itemize}
	
\textbf{Art.2 d.p.r. 318 Principi generali} 

L'installazione, l'esercizio e la fornitura di reti di telecomunicazioni nonché la prestazione dei servizi ad esse relativi accessibili al pubblico sono attività di preminente interesse generale, il cui espletamento si fonda: 
\begin{enumerate}
    \item sulla libera concorrenza e pluralità dei soggetti operatori, nel rispetto dei principi di obiettività, trasparenza, non discriminazione e proporzionalità; 
    \item sul rispetto degli obblighi di fornitura del servizio universale; 
    \item sulla tutela degli utenti e sulla loro libertà di scelta tra i servizi forniti dai diversi operatori;
    \item sull'uso efficiente delle risorse;
    \item sulla tutela dei diritti e delle libertà fondamentali, nonché dei diritti di persone giuridiche, enti o associazioni, in particolare del diritto alla riservatezza per quanto riguarda il trattamento dei dati personali nel settore delle telecomunicazioni;
    \item sul rispetto della vigente normativa in materia di tutela della salute pubblica, dell'ambiente e degli obiettivi di pianificazione urbanistica e territoriale, di concerto con le competenti autorità;
    \item sullo sviluppo della ricerca scientifica e tecnica, anche al fine di favorire la formazione in materia di telecomunicazioni. 
\end{enumerate}

\textbf{Art.6 d.p.r. 318 Autorizzazioni generali e licenze individuali }
\dots (Vedi slide)

