\subsection{Caratteri di fondo della legge 249/97}

Introduce l'enorme novità di un'Autorità garante: la differenza viene fatta a livello di entità dell'autorità e di poteri ad essa affidata, che non hanno precedenti nel panorama giuridico italiano.

L'estensore della legge è l'allora ministro delle Poste e Telecomunicazioni Antonio Maccanico.

Si tratta di una legge che implementa e incarna i processi di convergenza giuridica e comprende in un'unica normativa segmenti dell'ambito delle tlc precedentemente normati da legislazioni differenti e talvolta radicalmente separate.

Se le discipline precedenti (soprattutto quelle sulla televisione) erano dettagliate e molto vincolanti, AGCOM viene definita in modo molto meno dettagliato e molto più ampio. 
A questa Autorità viene data la facoltà di gestire lo sviluppo del settore anche in funzione delle innovazioni tecnologiche. 

Si cerca dunque di recepire lo spirito comunitario e migliorare l'efficienza legislativa lavorando in modo indipendente dal potere politico. 

Questo tipo di tecnica legislativa si diffonderà poi nel sistema legislativo italiano per tutti quei settori di difficile gestione o che si occupino di materie difficili e delicate. 


\subsubsection{Istituzione dell'AGCOM}
AGCOM è caratterizzato da struttura collegiale:
\begin{itemize}
    \item Presidente: nominato dal Capo dello Stato
    \item Commissione per le infrastrutture e reti
    \item Commissione per i servizi e prodotti
    \item Consiglio: composto dall'unione dei primi tre organi
\end{itemize}

Questa autorità dura 7 anni, l'attuale presidente è Giacomo Lasorella. 

Poteri di governo attribuiti all'AGCOM:
\begin{itemize}
    \item Competenze dell'ex Garante per la radiodiffusione e l'editoria
    \item Poteri consultivi e di proposta:
    \begin{itemize}
        \item offre un parere sullo schema del piano nazionale di ripartizione delle frequenze (a livello internazionale l'organismo che pianifica l'utilizzo dello spettro radio è l'ITU che si occupa di allocare lo spettro a diversi servizi, attribuire canali radio a diversi paesi, gestire fasi negoziali da parte dei Paesi. In ogni Stato si indicano poi le frequenze utilizzabili dai servizi di tlc e la collocazione degli impianti, delle aree di servizio e delle frequenze assegnate agli impianti).
        \item proporre al Ministro di settore lo schema di convenzione annessa la concessione del servizio pubblico radiotelevisivo
        \item proporre al Ministro i disciplinari per il rilascio delle concessioni e delle autorizzazioni in materia radiotelevisiva
        \item avanzare proposte riguardo la gestione delle innovazioni tecnologiche
    \end{itemize}
    \item Poteri di regolazione e di controllo:
    \begin{itemize}
        \item Approvare piani di assegnazione delle frequenze
        \item Definire le misure di sicurezza nelle comunicazioni
        \item Definire i criteri per la determinazione delle tariffe massime
        \item Determinare i criteri di definizione dei piani di numerazione nazionale delle reti
        \item Emanare direttive sui livelli di qualità dei servizi
        \item Adottare regolamenti sulla propria organizzazione e funzionamento
        \item Tenere il registro degli operatori di comunicazione
        \item Vigilare sul rispetto di obblighi e divieti previsti dalla legge
    \end{itemize}
    \item Poteri paragiurisdizionali e sanzionatori:
    \begin{itemize}
        \item Dirimere le controversie tra operatori
        \item Dirimere le controversie tra organismi di TLC e utenti
        \item Istruire le questioni attinenti a presunte violazioni dei limiti antitrust
        \item Imporre sanzioni pecuniarie a operatori
    \end{itemize}
    \item Poteri autorizzatori e relativi alla fornitura del S.U.
\end{itemize}

\subsubsection{I CORECOM - Comitati regionali per le comunicazioni}
Previsti dall'art. 1 c.13 della l.249/97; istituiti tramite leggi regionali e province autonome derivanti da criteri generali stabiliti dall'intesa tra AGCOM e conferenza Stato-Regioni.

Ogni comitato regionale per le comunicazioni è dotato di 5 membri, nominati dal consiglio regionale.

Sono concepiti come organi regionali ma dipendenti funzionalmente dall'Autorità per l'esercizio su scala regionale delle funzioni di governo, vigilanza e controllo.

\subsubsection{Il Consiglio Nazionale degli Utenti}
Organo composto da esperti designati dalle associazioni rappresentative delle varie categorie degli utenti dei servizi di TLC e radiotelevisivi. 

Il CNU è composto da 11 membri che eleggono presidente e vicepresidente. 

Il comitato ha poteri sia consultivi che di proposta rispetto a soggetti istituzionali e non (AGCOM, Parlamento e Governo). Non è prevista facoltà di intervento diretto.

\subsubsection{Rapporti tra AGCOM e Autorità per la concorrenza}
All'AGCM (Autorità garante della concorrenza e del mercato) compete l'accertamento e la repressione di comportamenti lesivi della concorrenza quando posti in essere da imprese che operano nel settore delle TLC.
AGCM è tenuto a esprimere pareri nei confronti dell'AGCOM (su provvedimenti sanzionatori) e l'AGCOM deve effettuare obbligatoriamente segnalazioni nei confronti dell'AGCM (su violazioni della normativa antitrust).

\subsubsection{La legge Antitrust}
Il limite massimo dell'indicatore di penetrazione in diversi segmenti del mercato delle TLC è posto al 25\%. 
Sono inoltre vigenti obblighi di trasparenza alle imprese per consentire il rispetto del limite e si ritiene che la fissazione di un limite \textit{ex-ante} sia più idonea a favorire l'avvio di un mercato aperto e concorrenziale.