\subsection{La Review '99}

Se la prima serie di direttive aveva l'obiettivo di avviare l'apertura dei mercati, nel '99 si rese necessario rettificare alcune rigidità introdotte in una fase iniziale.

A questo fine fu approvato dalla Commissione Europea un documento noto come ``The 1999 Communications Review" che esaminava il quadro normativo delle telecomunicazioni. 
In questo documento si analizzano le linee portanti che la CE intendeva perseguire:
\begin{itemize}
    \item Ricomporre e armonizzare l'abbondante normativa comunitaria degli anni precedenti in materia di TLC
    \item Porre le basi per la futura regolamentazione del settore
\end{itemize}

L'obiettivo finale di questo documento era adeguare il contenuto della normativa vigente in modo da guidare nel modo corretto lo sviluppo delle TLC.


I tre obiettivi da perseguire erano:
\begin{enumerate}
    \item promozione di un mercato Europeo aperto e competitivo nell'interesse e tutela del consumatore (rapporto ottimale qualità-prezzo).
    \item promozione di un effettivo accesso di tutti i cittadini europei a beneficio del servizio universale, alla tutela di riservatezza dei dati personali e alla garanzia di trasparenza delle tariffe sui servizi.
    \item consolidamento del mercato interno alla luce del fenomeno della convergenza mediante la rimozione degli ostacoli alla fornitura di reti e servizi a livello europeo.
\end{enumerate}

Essi erano perseguibili operando a favore di un'unificazione e alleggerimento delle regole a livello internazionale, in particolare ricorrendo ad atti di ``soft-law" (orientamenti e raccomandazioni) anziché atti di normazione primaria.

