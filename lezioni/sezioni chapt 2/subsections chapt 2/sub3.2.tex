\subsection{La direttiva quadro (2002/21/CE)}

La direttiva contiene tre novità rilevanti:
\begin{enumerate}
    \item Fissazione di principi comuni per l'intero settore della comunicazione elettronica, superando la distinzione tra telecomunicazioni e radiotelevisione (traduzione in termini normativi di un fenomeno di convergenza di tecnologie e mezzi di comunicazione)
    \item Diversa filosofia per orientare l'armonizzazione delle legislazioni nazionali in materia, ispirata ad una progressiva sostituzione delle regolazioni ex-ante con regolazioni ex-post, più flessibili
    \item Ruolo delle Autorità nazionali di regolamentazione (ANR), loro reciproci rapporti, loro rapporti con la Commissione Europea (novità che incide sugli aspetti istituzionali dell'intero sistema)
\end{enumerate}

Le autorità nazionali sono concepite come organismi tecnici e indipendenti necessari per garantire imparzialità e trasparenza. 
Esse portano avanti attività orientate alla promozione della concorrenza nelle reti e nei servizi, allo sviluppo del mercato interno e alla tutela dei diritti dei cittadini dell'Unione.

A tal fine fu costituito un gruppo di regolatori europei ERG formato dalle autorità nazionali con ruolo consultivo nei confronti della Commissione.

In tal modo si ridusse a un ruolo marginale il compito dei legislatori e degli esecutivi nazionali, in quanto ad essi viene concesso un margine di discrezionalità ridotto.