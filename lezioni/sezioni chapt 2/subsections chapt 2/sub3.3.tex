\subsection{La direttiva autorizzazioni (2002/20/CE)}

Le precedenti direttive individuano due strumenti (autorizzazioni generali e licenze individuali) ma lasciano la libertà agli Stati membri di scegliere quali strumenti privilegiare. 

In moltissimi Stati prevalse l'uso delle licenze individuali, con la conseguenza di produrre ostacoli rispetto all'accesso al mercato da parte di nuovi fornitori di servizi.


La nuova direttiva autorizzazioni fissò il principio per cui in futuro la fornitura di reti e servizi sarebbe dovuta essere assoggettata \textit{solo} a un'autorizzazione generale.