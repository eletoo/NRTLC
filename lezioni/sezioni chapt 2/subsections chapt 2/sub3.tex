\subsection{Le prime direttive comunitarie: tre fronti di intervento}

Sviluppo tecnologico e grande evoluzione dei ruoli della comunità europea implicano la necessità di una gestione politica della complessità economica. I tre fronti di intervento principali sono:
\begin{enumerate}
    \item Prime parziali liberalizzazioni
    \item Liberalizzazione delle infrastrutture e dei servizi
    \item Interconnessione delle reti e servizio universale 
\end{enumerate}

Questi tre fronti si sviluppano in un primo corpo di direttive comunitarie finalizzato al passaggio dal monopolio alla liberalizzazione del mercato comunitario.

\subsubsection{Prime parziali liberalizzazioni}
\begin{itemize}
    \item Direttiva sui terminali: garantire la concorrenza sui mercati dei terminali di telecomunicazioni, eliminazione dei diritti esclusivi di importazione, commercializzazione e manutenzione dei terminali
    \item Direttiva sulle reti (\textit{Open network provision}): armonizzazione delle condizioni per l'accesso e l'uso libero efficace delle reti pubbliche; in sostanza altri operatori devono poter garantire i propri servizi eventualmente attraverso la rete pubblica senza essere ostacolati
    \item Direttiva sui servizi: liberalizzazione dei servizi di telecomunicazioni a valore aggiunto, stabilisce l'obbligo per gli ex-monopolisti di consentire l'accesso alle proprie reti ai fornitori di servizi liberalizzati. Stabilisce condizioni di accesso eque al mercato concorrenziale (cioè permettere ad altri operatori di accedere senza guadagnarne eccessivamente ma senza nemmeno permettere a chiunque di entrare: introduzione di barriere di accesso al mercato)
\end{itemize}

\subsubsection{Liberalizzazione delle infrastrutture e dei servizi}

Circa 5 anni dopo le prime parziali liberalizzazioni. \
\begin{itemize}
    \item Liberalizzazione delle infrastrutture: si supera la parziale conservazione dei monopoli impostata nel libro verde. Si elimina alcune restrizioni sull'uso di reti televisive via cavo per la fornitura di servizi di tlc già liberalizzati. Viene consentito l'impiego di altre reti cablate per servizi di TLC diversi dai servizi di telefonia vocale e si autorizza l'interconnessione di tali reti alla rete pubblica di TLC. Completa apertura alla concorrenza (\textit{full competition}).
    \item Liberalizzazione dei servizi: armonizzazione delle legislazioni nazionali in merito alla disciplina dei titoli abilitativi
\end{itemize}

\subsubsection{Interconnessione delle reti e servizio universale}

Open network provision: interconnettività a livello logico e fisico tra le reti di tlc in modo che non ci siano incompatibilità.

Nella telefonia c'è stata una fase in cui valeva una parziale interconnessione ma permaneva un problema relativo agli operatori fornitori di servizi (es. per chiamare qualcuno con un operatore diverso era a carico dell'utente l'inserimento di un prefisso opportuno per segnalare la chiamata tra operatori differenti, oppure per cambiare operatore bisognava anche cambiare numero di telefono). 

La politica dell'UE non è quella di espropriare gli stati di un asset importante dal punto di vista economico ma quello di incentivare a privatizzare le aziende e garantire una regolamentazione per gli operatori (dando più diritti ai nuovi entranti nel mercato e più doveri agli ex-monopolisti). 

L'ultima barriera di mercato abbattuta di recente sono le tariffe di roaming per le chiamate internazionali.


Uno dei capisaldi che giustificavano il monopolio era il fatto che si trattasse di servizi che lo Stato doveva garantire a tutti i cittadini indipendentemente dalla posizione geografica agli stessi prezzi in virtù di un principio di equità sociale.

Il principio di equità sociale va coniugato con il contesto in cui i monopoli sono stati distrutti, quindi ecco apparire il concetto di servizio universale.

Servizio universale: insieme minimo definito di servizi, di una data qualità, a disposizione di tu; gli utenti, indipendentemente dalla localizzazione geografica e offerto, in funzione delle specifiche condizioni nazionali, ad un prezzo abbordabile.

Se il servizio universale non porta ricavi sufficienti a compensare la spesa sostenuta per garantirlo, esso viene finanziato attraverso le tasse dei cittadini oppure attraverso un fondo alimentato dagli operatori (secondo regole fissate dagli Stati singolarmente).