\subsection{Il codice delle Comunicazioni Elettroniche}

Il Codice delle Comunicazioni Elettroniche (d.lgs. 259/2003) propone un’attuazione della nuova normativa
comunitaria tuttavia se ne discosta per alcuni aspetti rilevanti:
\begin{itemize}
    \item Mantiene la distinzione tra le TLC e la radiotelevisione. Questo viene confermato all’iniziativa legislativa che porterà al Testo Unico della Radiotelevisione (d.lgs. 177/2005); tutto questo porta ad ulteriori scostamenti dalle direttive.
    \item Differisce temporaneamente il nuovo regime dei titoli abilitativi (autorizzazione generale). Fa delle autorizzazioni generali la regola e delle licenze individuali l’eccezione, tuttavia mantiene invariate le scadenze dei titoli precedenti alla nuova normativa (in contrasto con la direttiva stessa che invece anticipa le scadenze).
    \item Mantiene rilevanti poteri al Ministero di settore → Anche se l’AGCOM viene riconosciuto come autorità nazionale di regolamentazione, vengono attribuiti molto poteri rilevanti al Ministero di settore; questo va contro la necessità di separazione per avere garanzia di indipendenza e imparzialità.
\end{itemize}
