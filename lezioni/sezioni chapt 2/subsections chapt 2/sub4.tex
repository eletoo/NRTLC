\subsection{Caratteristiche di fondo della prima normativa comunitaria}

La politica europea richiede un equilibrio tra la componente politica in sé e la richiesta di regolamentare un ambito (il mondo e il mercato digitali) che dovrebbe in linea teorica rimanere libero da influssi politici, pertanto si è reso necessario applicare un ragionamento orientato alla previsione di possibili sviluppi futuri del mercato digitale.

Se da un lato vediamo aspetti positivi nell'applicazione di una normativa comunitaria (es. privatizzazione dei monopoli statali in virtù di una libertà di mercato) dall'altro si possono verificare casi di eccessiva liberalizzazione. 

Abbiamo visto nella scorsa lezione che sono state introdotte leggi per la liberalizzazione con garanzia di concorrenza, in modo da privatizzare i monopoli statali impedendo che si verifichi un monopolio privato, garantendo dunque l'esistenza della concorrenza. 
Non c'è un'espropriazione dell'infrastruttura di rete in sé, ma si concede che altri soggetti possano contribuire alla realizzazione e gestione di altre reti da introdurre in quella già esistente. 
Lo strumento per riuscire a garantire la concorrenza è la regolamentazione asimmetrica che valuta quali sono gli operatori a più grossa forza di mercato e impone degli obblighi specifici a tali operatori. Per applicare la regolamentazione asimmetrica è necessario avere conoscenza della struttura del mercato di interesse, quindi diventano fondamentali le analisi di mercato. 

Abbiamo visto inoltre il principio del servizio universale, ossia il trasferimento dal monopolista statale alle aziende private dell'obbligo di garantire a tutti i cittadini un servizio minimo. In qualsiasi momento storico ci sarà un insieme di servizi fondamentale per garantire la cittadinanza attiva e, nel caso delle TLC, una vita di relazioni per i cittadini. 

Il servizio universale non necessariamente trova attenzione in un sistema di libero mercato perché da esso non si ricava profitto diretto. Come servizio prevede un meccanismo per cui tutti gli operatori contribuiscono alla garanzia del servizio stesso (eventualmente avvantaggiati da un contributo statale creato attraverso le tasse dei cittadini) attraverso un fondo finalizzato a questo scopo. 

I pubblici poteri diventano un soggetto regolatore del settore supportati dalle normative; la CE ritiene che sia cosa buona che il settore delle TLC sia gestito da autorità indipendenti dalla politica nazionale. In Italia si tratta dell'AGCOM.