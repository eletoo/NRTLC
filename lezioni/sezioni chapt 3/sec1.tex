\section{Radiotelevisione: passato remoto ma non troppo}

Il valore più importante in materia di radiotelevisione è senza dubbio il principio pluralistico, come per le TLC lo era la convergenza tecnologica; si divide in:
\begin{itemize}
    \item Pluralismo esterno, che consiste nell’offrire ai cittadini una possibilità di scelta tra più fonti informative in relazione anche alla disponibilità dei mezzi tecnici necessari.
    \item Pluralismo interno, ossia l’obbligo per il servizio pubblico di dar voce a tutte le opinioni e correnti di pensiero (par condicio).
\end{itemize}

Per il cittadino questo si traduce in concreta possibilità di scelta tra diverse fonti informative.

Particolarmente importanti risultano le due sentenze della Corte Costituzionale del 1974 con cui vengono
dichiarate illegittime la riserva allo Stato dell'attività di ritrasmissione delle emittenti estere e la riserva allo
Stato nel settore della radiotelevisione via cavo. Le TV estere ritrasmesse in Italia erano TeleMonteCarlo (oggi
trasformata in La7), TV Koper-Capodistria e la TV della Svizzera Italiana.\bigskip

La prima legge di sistema è stata la Legge riforma n.103/1975; questa stabilisce che la radio e telediffusione sono due servizi pubblici essenziali a cui viene garantita libertà d’espressione; in particolare il servizio pubblico radiotelevisivo è riservato allo Stato e viene affidato ad una società a totale partecipazione pubblica in regime di concessione (con atti di durata di 6 anni rinnovabili); invece per attività di gestione degli impianti ripetitori,
di installazione ed esercizio di impianti di trasmissione via cavo (locale) e di filodiffusione è previsto un regime autorizzatorio (con autorizzazioni a livello locale o regionale).


Con una nuova sentenza della Corte Costituzionale n.202/1976 viene dichiarata illegittima la riserva allo
Stato dell’intero settore della radiotelediffusione a livello locale, aprendo così la strada ai privati in questo
ambito (non solo in ambito di TV via cavo, come detto nel '74); quindi la C.C. difende l’iniziativa privata, ma il problema è che le frequenze sono una risorsa scarsa e quindi si dà origine ad una “competizione” per le frequenze (per questo si fa grande differenza tra locale e nazionale). A questa sentenza tuttavia non segue alcuna legge che disciplini la comunicazione in etere fino al 1990, si parla del “far west dell’etere”. Intervengono negli anni successivi gruppi editoriali e nel 1978 Berlusconi vara Tele Milano 58; nel 1980 nasce Canale5 con la trasmissione in contemporanea dello stesso programma da parte di diverse emittenti; nel 1982 iniziano a trasmettere anche Italia1 (Rusconi) e Rete4 (Mondadori), che vengono poi acquisite da Berlusconi.


Nei primi anni '80 la C.C. “premeva” molto per l’apertura del mercato, ma si premurava che non diventasse un “duopolio” pubblico-privato”; in particolare in attesa di una legge più ampia il governo intervenne in favore dei network privati: “asse Craxi-Berlusconi”.

A seguito della conferenza dell’ITU del 1984 fu approvato un decreto legge (d.l. n. 807/1984) che stabilì che entro il luglio 1987 tutti gli Stati dovessero effettuare la mappatura delle frequenze; lo Stato italiano non sapeva quali e quante emittenti private e quali e quanti impianti di radiodiffusione fossero attivi nel paese (far west dell’etere). Dopo il censimento si manifestò la necessità di effettuare un riordino in quanto all’Italia corrispondeva solo circa il 20\% di quelle effettivamente usate. Sempre questo d.l. stabilisce che la diffusione sonora e televisiva sull’intero territorio nazionale è riservata allo Stato, che il servizio pubblico radiotelevisivo è esercitato dallo Stato mediante concessione ad una società per azioni a totale partecipazione pubblica; solo in una “futura” legge si consentirà l’iniziativa privata predisponendo norme antitrust, in attesa di questa restano consentite le trasmissioni contemporanee degli stessi programmi preregistrati (per questo è detto decreto “salva Berlusconi”).\bigskip


Si ha un’ulteriore Sentenza della Corte Costituzionale (n. 826) del 1988 che chiarisce il concetto di pluralismo dell’informazione come la possibilità di ingresso nell’emittenza pubblica e privata di quante più voci consentano i mezzi e come la possibilità di esprimere le proprie opinioni senza pericolo di essere emarginati; tuttavia la C. C. ha poi “salvato” la disciplina transitoria proprio perché tale, sancendo che ne occorre una definitiva capace di ostacolare la concentrazione di monopoli/oligopoli con un’effettiva tutela del pluralismo dell’informazione.


La Direttiva 89/552/CE stabilisce il principio di libera circolazione delle trasmissioni di paesi comunitari, obbliga la trasmissione di opere europee, pone dei limiti sulla trasmissione di opere cinematografiche e disciplina pubblicità e sponsorizzazioni.