\section{La legge Mammì e le sue conseguenze}

Legge n. 223/1990, Disciplina del sistema radiotelevisivo pubblico e privato (Legge Mammì), dopo oltre 5
anni di transitorio; è molto critica perché “chiude il transitorio fotografando una situazione di fatto”. Essa
regola:
\begin{itemize}
    \item Principi generali comuni → Di fondamentale importanza risultano il pluralismo, l’obiettività,l’imparzialità dell’informazione, l’apertura alle diverse opinioni e il rispetto delle libertà.
    \item Pianificazione delle radiofrequenze → Effettuata mediante un piano nazionale di ripartizione ed assegnazione; il territorio viene suddiviso in bacini di utenza i quali devono consentire coesistenza del maggior numero possibile di impianti ed una adeguata pluralità di emittenti e reti.
    \item Regime concessorio per tutte le emittenti → Il servizio pubblico radiotelevisivo è affidato mediante concessione ad una società per azioni a totale partecipazione pubblica; possono inoltre essere rilasciate concessioni a privati per radiodiffusione sonore sia locale che nazionale ed anche per telediffusione in ambito locale di singole emittenti e reti (durata di 6 anni rinnovabili, con minimo sul capitale).
    \item Normativa antitrust → Vengono vietate le posizioni dominanti negli ambiti dei mezzi di comunicazione di massa, in particolare non è possibile essere contemporaneamente titolari di concessioni sia locali che nazionali, di più di 3 concessioni a livello nazionale o di più di una a livello di singolo bacino d’utenza.
    \item Disciplina della pubblicità
    \item Strumenti di garanzia per una corretta applicazione
\end{itemize}

Viene poi istituito il Garante per la radiodiffusione e l’editoria che durava in carica 5 anni e svolgeva diverse attività in quest’ambito (sostituito nel 1997 dall’AGCOM); si ha poi la delega al governo a provvedere alla disciplina della TV via cavo, distinguendo tra attività di installazione e gestione di impianti (riserva allo Stato) e attività di distribuzione di programmi attraverso tali impianti (nessuna riserva allo Stato).


Il d.l. 323/1993, convertito poi in legge, stabilisce che l’etere debba essere libero: infatti le pay-tv dovranno essere trasferite su cavo o satellite entro un termine stabilito; la legge Mammì non diceva nulla a riguardo delle pay-TV, ma anche la direttiva europea ne ammetteva la possibilità di esistenza.


La Legge Mammì introduce la normativa antitrust in una situazione di totale anarchia, ma non risolve il duopolio RAI-Fininvest (Berlusconi); varie emittenti locali hanno fatto ricorso poiché, pur essendo in graduatoria, non avevano ottenuto la concessione oppure non avevano ottenuto frequenze sufficienti; di conseguenza, la C.C. ha dichiarato irragionevole e incostituzionale il limite antitrust (3 reti o max il 25\% delle
frequenze) poiché non idoneo per prevenire la formazione di posizioni dominanti; tuttavia ancora una volta rimane inascoltata.

La “tecnica” prescelta dalla C.C. dal 1984 al 1994, cioè quella di “salvare” il transitorio in attesa di una legge organica di riforma, ha portato al fatto che, quando il legislatore è effettivamente intervenuto con la legge 223/1990, la situazione fosse tale da non poter essere modificata (e rimarrà sostanzialmente uguale fino ai giorni nostri).
