\section{La legge Maccanico}

La legge 249/1997 (legge Maccanico, legge di sistema per le TLC) istituisce l’AGCOM al fine di rendere più efficaci i limiti antitrust, ma con scarso successo, infatti l’Autorità è stata più attiva nel settore delle TLC che in quello RTV. I nuovi limiti antitrust stabiliti causano l’eccedenza di una rete RAI e di una rete Mediaset (i limiti vengono abbassati da 25\% a 20\%), inoltre nessun singolo operante a livello nazionale può detenere più del 30\% delle risorse economiche complessive del settore.


Come la TV via cavo anche quella via satellite diviene soggetta ad un regime autorizzatorio e sottoposta ai predetti limiti antitrust; inoltre si consente alla concessionaria del servizio pubblico radiotelevisivo e di TLC la possibilità di partecipare ad un’unica piattaforma digitale (unico decoder per tutte le tipologie di trasmissione (via cavo, via satellite e analogiche)), con l’intento di favorire la nascita di un unico grande operatore di televisione digitale con capitale a maggioranza italiana: tuttavia le trattative tra RAI e Telecom non sono andate a buon fine e si sono mantenute due distinte piattaforme satellitari: RAI-Telepiù e Stream. 


Viene poi stabilito che dal primo gennaio 2000 la fruibilità dei diversi programmi digitali con accesso condizionato e la ricezione dei programmi radiotelevisivi digitali in chiaro dovesse essere obbligatoriamente possibile mediante un unico apparato decodificatore. Nel 2002 Rupert Murdoch, azionista di Stream acquista Telepiù dando vita a SKY; la fusione è stata permessa dalla CE a patto che venisse garantito da parte di SKY l’accesso alla propria piattaforma ad altri eventuali concorrenti satellitari, inoltre fino al 2011 SKY si impegnerà a non intraprendere alcuna attività nel settore
della TV digitale terrestre.