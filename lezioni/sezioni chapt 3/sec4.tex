\section{La legge delega (Gasparri) e il testo unico sulla radiotelevisione}

Si ha un ulteriore periodo di transitorio dopo la Legge Maccanico, al termine del quale sarebbero entrati in vigore i limiti antitrust previsti e sarebbe stato approvato il piano nazionale di assegnazione delle frequenze; tuttavia l’AGCOM non attende la liberazione delle frequenze eccedenti (vedi Rete4) e ha elaborato comunque il piano di assegnazione delle frequenze. Si hanno inoltre reiterate proroghe del termine di assegnazione delle frequenze e nel 2001 si avviano le sperimentazioni per la trasmissione in digitale che dovrebbero terminare nel 2006. Il termine sembrava essere fissato per la fine del 2003, tuttavia entro tale data nemmeno un quarto degli utenti aveva avuto accesso al digitale terrestre, quindi sembrava imminente una nuova proroga da parte dell’AGCOM; questa è stata tuttavia dichiarata illegittima dalla C.C.; urgeva quindi una nuova legge di sistema.


In ottica della nuova legge di sistema, nel luglio del 2002 il Presidente della Repubblica Ciampi richiama i parlamentari a porre l’attenzione sull’importanza dei valori del pluralismo e dell’imparzialità dell’informazione. Con la Sentenza della Corte Costituzionale n.466 del 2002 viene dichiarata incostituzionale la legge 249/1997, che permette all’AGCOM di prorogare indefinitamente il termine del periodo di transitorio (non è criticato il transitorio in sé, ma la sua durata).


Nel settembre del 2002 cominciano i lavori per il disegno di legge governativo sulla riforma del sistema radiotelevisivo che vengono terminati in poco tempo (a seguito della sentenza della C.C.). Tuttavia l’entrata in vigore viene annullata dal Presidente della Repubblica e nel frattempo viene emanato un decreto che consentiva alle reti eccedenti di continuare la programmazione (decreto “salva Rete4”) e all’AGCOM di continuare la sua analisi sullo sviluppo del digitale terrestre. Tale legge viene infatti considerata incostituzionale poiché si potrebbero formare delle posizioni dominanti, mancano seri limiti alla raccolta pubblicitaria e non viene precisato cosa accadrebbe se l’AGCOM dovesse accertare che non sussistono sufficienti condizioni di sviluppo del digitale terrestre.


Quindi il testo della legge viene riesaminato dalle Camere, parzialmente modificato e riapprovato: Legge 3 Maggio 2004 n.112 (Legge Gasparri), che tra le altre cose delega al Governo l’emanazione del Testo Unico sulla Radiotelevisione.

Con il Testo Unico della Radiotelevisione (d. lgs. 177/2005) risulta abrogata gran parte della legge del 1975, della legge Mammì e della disciplina antitrust della legge Maccanico; il testo contiene:
\begin{itemize}
    \item I principi generali che informano l’assetto del sistema radiotelevisivo nazionale, regionale e locale e lo adeguano all’introduzione della tecnologia digitale ed al processo di convergenza tra la radiotelevisione ed altri settori TLC; e anche le disposizioni legislative in materia radiotelevisiva.
    \item Le disposizioni in materia di trasmissione dei programmi televisivi e radiofonici.
    \item Le definizioni dei soggetti della comunicazione, in particolare:
        \begin{itemize}
            \item Operatore di rete → Il titolare del diritto di installazione, esercizio e fornitura di una rete di comunicazione elettronica e di impianti di messa in onda.
            \item Fornitore di contenuti → Colui che ha la responsabilità editoriale nella predisposizione dei programmi televisivi o radiofonici.
            \item Fornitore di servizi → Colui che fornisce servizi al pubblico (attraverso l’operatore di rete). Possono essere ricoperti da uno stesso soggetto senza limiti, ma con clausole a garanzia di pluralismo e concorrenza (titoli abilitativi distinti, obbligo di separazione contabile, di non discriminazione, …).
        \end{itemize}
    \item I principi fondamentali del sistema radiotelevisivo che riguardano la libertà e il pluralismo dei servizi offerti, la tutela della concorrenza nel mercato radiotelevisivo e dei mezzi di comunicazione di massa contro comportamenti lesivi del pluralismo, la definizione delle norme antitrust, gli obblighi per gli operatori di rete e i fornitori di contenuti, l’obbligo di separazione contabile.
    \item Il fatto che è l’AGCOM a dover verificare che il servizio pubblico radiotelevisivo venga effettivamente prestato ai sensi delle disposizioni vigenti, con possibilità di indagini e di sanzioni.
    \item Le disposizioni riguardo alla pianificazione delle frequenze; in particolare l’assegnazione delle frequenze avviene secondo criteri pubblici, obiettivi, trasparenti, non discriminatori e proporzionati attraverso un piano adottato dal Ministero, approvato dal decreto del Ministro e aggiornato ogni 5 anni.
\end{itemize}

Il T.U. introduce il Sistema Integrato delle Comunicazioni (SIC) ossia il settore economico che comprende stampa, editoria, radio e televisione, cinema, pubblicità e sponsorizzazioni; è un enorme paniere che funge da parametro di riferimento per calcolare i nuovi limiti antitrust; in particolare il limite del 30\% fissato nel 1997 era calcolato su 12 miliardi, ora viene abbassato al 20\% ma su un totale di 26 miliardi, quindi il valore assoluto risulta più alto (“una furbata”), quindi la legge tende a “salvare il passato” e risulta inefficace nell’eliminare le posizioni dominanti già presenti; i soggetti del SIC devono notificare l’AGCOM tutte le situazioni che potrebbero portare a posizioni dominanti e il procedimento che attua l’AGCOM in caso di rischio di superamento dei limiti è più o meno simile al precedente.


Un caso eclatante è quello dell’emittente Centro Europa 7. Nasce come network di reti locali e pur ottenendo la concessione non riesce a trasmettere in quanto le frequenze non erano sufficienti (erano ancora occupate dalle reti eccedenti). Nonostante la sentenza 466/2002 l’emittente continua a non trasmettere e nessun governo prende provvedimenti per risolvere il problema (vedi decreto “salva Rete4”). Europa7 fa quindi ricorso e chiede un risarcimento allo Stato per mancata attività. La Corte di Giustizia dell’UE si esprime a riguardo nel 2008 invitando il Consiglio di Stato a togliere le frequenze a Rete4 e assegnarle ad Europa7; tuttavia il governo Berlusconi per “aggirare” questa sentenza emana un d.l. che stabilisce che la prosecuzione dell’esercizio degli impianti di trasmissione è consentita a tutti quelli che ne hanno titolo fino alla scadenza prevista, cioè fino al 2012.


Il Consiglio di Stato ha quindi respinto il ricorso (che puntava all’annullamento della concessione a Rete4), considerandolo tardivo, così come ha ritenuto inammissibile la richiesta dell’emittente di condannare il Ministero dello Sviluppo Economico; inoltre, il risarcimento proposto non è stato quello effettivamente richiesto e Europa7 non ha ottenuto l’assegnazione delle frequenze. Solo nel dicembre 2008 Europa7 ha ottenuto il canale 8 da usare in analogico e/o digitale, ritenuta però un’assegnazione insufficiente per via della copertura di territorio ridotta.

La disputa si è risolta definitivamente nel 2010, assegnando ad Europa7 ulteriori canali, ma solo in seguito all’intervento della Corte Europea dei Diritti dell’Uomo; quindi dopo circa 12 anni dalla concessione del diritto a trasmettere, si è avuto l’inizio effettivo delle trasmissioni (Europa7 HD).