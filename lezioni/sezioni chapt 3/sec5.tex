\section{La disciplina dell’attività televisiva in digitale terrestre}

Nel 2001 si avvia la fase sperimentale della “TV digitale terrestre”, con fine prevista entro il 2006, dopodiché la tecnica analogica sarebbe stata del tutto abbandonata (switch off); questa tecnica avrebbe permesso non più un solo programma per ciascun canale, ma da 4 a 7 programmi simultaneamente, per ogni multiplex, così si sarebbe raggiunto il pluralismo invocato dalla C.C. e dalla U.E., inoltre la qualità dell’immagine sarebbe stata migliore e sarebbe diminuito l’inquinamento elettromagnetico. A tal fine i soggetti che usano legittimamente frequenze terrestri per la radiodiffusione vengono abilitati alla sperimentazione del digitale terrestre.


Il regolamento per il digitale terrestre (DVB-T) sancisce la separazione tra operatore di rete, fornitore di contenuti e fornitore di servizi, e vengono anche stabilite le modalità per il rilascio delle licenze a questi. Nel luglio 2006 la Commissione europea invia all’Italia una lettera di costituzione in mora invitandola a emendare la legge Gasparri per allinearla alle prescrizioni europee del 2002, infatti c’era il rischio che in Italia venissero favoriti gli operatori già operanti (RAI e Mediaset in particolare) a scapito dei newcomers; l’Italia rimane però inosservante e quindi la Commissione si ritrova costretta ad invitarla nuovamente a prendere le disposizioni necessarie entro due mesi (non rispettato). Il problema stava nell’impossibilità di individuare mercati rilevanti e barriere all’accesso a questi e dunque serviva una nuova legge di sistema che però è stata sospesa per via delle delibere AGCOM del 2007 sul passaggio al digitale terrestre.


La Delibera AGCOM 645/07/CONS stabilisce la suddivisione delle 21 reti nazionali in 8 per la conversione delle attuali reti analogiche, 8 per la conversione in singola frequenza delle attuali reti digitali esistenti che usano la multifrequenza (meno efficace); alla fine di questo saranno disponibili 5 reti (digital dividend), 3 riservate ai nuovi entranti e 2 aperte a chiunque ma con un limite di 5 multiplex per ciascuno. Sono inoltre previste misure asimmetriche per aumentare la concorrenza. 


La gara di assegnazione delle frequenze è una gara non competitiva e gratuita, detta quindi “Beauty Contest”; tuttavia i requisiti per parteciparvi vengono stabiliti dal Ministro dello sviluppo economico, con una conseguente scarsa imparzialità. Tuttavia nel 2012 il Ministro dello sviluppo economico, in accordo con Presidente del Consiglio Monti, decide di annullare il beauty contest, l’assegnazione avverrà con un’asta pubblica onerosa e saranno escluse da tutti i lotti RAI, Mediaset e TiMB. Nel 2014 all’asta per le frequenze presenta domanda un unico operatore, soltanto Cairo.


Nel 2005 la scadenza per lo switch off viene spostata dal 2006 al 2008, che rimane comunque una data irrealistica, ma in quell’anno si avrà la fissazione di un calendario per il passaggio definitivo al digitale terrestre, con termine nel 2012.