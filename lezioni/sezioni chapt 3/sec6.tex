\section{Le direttive “Media senza Frontiere” (2007, 2010)}

Ma cosa succede in Europa nel frattempo? Alla fine del 2007 viene adottata la Dir. 2007/65/CE riguardante servizi di media audiovisivi (SMAV) (“Media senza Frontiere”); gli Stati membri dovevano recepire il contenuto entro il 2009. Prevede una nuova definizione dei servizi audiovisivi svincolata dalle tecniche di trasmissione:
\begin{itemize}
    \item Servizi lineari → Televisione tradizionale, Internet e telefonia mobile.
    \item Servizi non lineari → I servizi di televisione a richiesta, quelli on-demand ad esempio.
\end{itemize}

Beneficiano tutti del principio del paese d’origine (applicazione extraterritoriale del diritto nazionale); così anche ai servizi non lineari si garantiscono le migliori condizioni per il successo commerciale. Inoltre, ogni emittente televisiva e produttore di opere cinematografiche può decidere come e quando interrompere i programmi gratuiti con la pubblicità (autoregolamentazione); rimane però il tetto orario del 20\% per pubblicità e televendite, viene autorizzato il product placement e vietato l’inserimento di prodotti durante notiziari, trasmissioni di attualità, documentari, programmi per bambini e pubblicità occulta. Ci sono poi obblighi per la protezione dei minori e di proibire contenuti che incitano all’odio.


Ci sono tre misure per la promozione del pluralismo dei media:
\begin{itemize}
    \item Obbligo di garantire l’indipendenza dell’autorità di regolamentazione nazionale.
    \item Diritto di utilizzare “brevi estratti” di cronaca di interesse pubblico.
    \item Promozione di contenuti audiovisivi prodotti da società europee indipendenti.
\end{itemize}

Inoltre, con il consolidamento della direttiva SMAV (SMAV2 n.13/2010), gli Stati membri non devono ostacolare la ritrasmissione dei servizi di media audiovisivi provenienti da altri Stati membri, a meno che non abbiano contenuti lesivi ai telespettatori.


Approvato nel marzo 2010, il decreto Romani modifica sensibilmente il T.U. sulla radiotelevisione, che si intitolerà T.U. sui servizi di media audiovisivi e radiofonici; con questo viene recepita la differenza tra servizi lineari e non, inoltre vengono introdotte nuove regole riguardanti il product placement, i limiti per gli spot pubblicitari e la trasmissione di opere cinematografiche e vietate ai minori.


I canali time shifted non si computeranno ai fini del rispetto del limite anticoncentrazione (in base al quale nessun operatore può controllare più del 20\% dei canali della televisione digitale terrestre). La pay per view viene menzionata come fornitore di servizi interattivi associati o di servizi di accesso condizionato (di fatto esce dalla disciplina dei servizi radio-TV). 

Da parte dell’AGCOM vengono emanati dei regolamenti riguardanti le autorizzazioni delle prestazioni di servizi media audiovisivi; viene esteso ai fornitori di servizi di media audiovisivi via Internet il regime autorizzatorio che di fatto consiste nel richiedere una specifica autorizzazione all’AGCOM per i fornitori di servizi in modalità lineare o inviare alla stessa una segnalazione certificata di inizio attività per i fornitori di servizi in modalità non lineare.