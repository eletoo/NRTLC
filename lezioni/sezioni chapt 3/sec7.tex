\section{Dicono di noi in Europa}

Il Parlamento europeo denota come il tasso di concentrazione del mercato televisivo in Italia sia il più elevato d’Europa; il gruppo Mediaset nel 2001 ha ottenuto i due terzi delle risorse pubblicitarie complessive, con il Presidente del Consiglio (Berlusconi) che non ha risolto il proprio conflitto di interessi: il sistema italiano presenta un’anomalia dovuta ad una combinazione unica di poteri economico, politico e mediatico nelle mani di un solo uomo ed opera quindi da decenni in assenza di legalità. 

L’OCSE, da parte sua, ha criticato il fatto che il livello di concorrenza nel sistema radiotelevisivo italiano resta ancora insufficiente perché il sistema è ancora dominato da società statali e da una società privata in posizione dominante.