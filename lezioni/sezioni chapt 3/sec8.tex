\section{Attualità del sistema radiotelevisivo}

Gli utenti vogliono poter accedere alle offerte audiovisive anytime, anywhere e any device (quadro ancora in evoluzione).
Nel 2016 in particolare, viene proposta una nuova direttiva che sostituisca o modifichi la direttiva SMAV, questo perchè il panorama dei media audiovisivi sta cambiando rapidamente per la sempre maggiore convergenza tra televisione e servizi via Internet: i consumatori chiedono sempre di più l’accesso a contenuti tramite smart TV e dispositivi portatili; emergono nuovi modelli imprenditoriali nel campo della radiodiffusione, in particolare vengono estesi i contenuti anche via Internet; inoltre, le azioni della UE sono sempre più improntate alla tutela dei consumatori, tramite azioni specifiche: a seguito di un’ulteriore modernizzazione della normativa, si vuole infatti unificare il mercato degli operatori tradizionali con quello degli Over-the-Top (level playing field).