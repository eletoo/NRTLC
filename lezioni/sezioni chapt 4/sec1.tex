\section{Quadro normativo del diritto d’autore}

La proprietà intellettuale può essere tutelata attraverso:
\begin{itemize}
    \item Diritto industriale, ossia i brevetti, i marchi, il diritto della pubblicità, ecc.; il brevetto tutela invenzioni, modelli di utilità, varietà vegetali, attraverso registrazione presso appositi uffici; per la tutelabilità devono esserci i requisiti di novità, liceità, attività inventiva e applicazione industriale; la tutela dura 20 anni dalla registrazione.
    \item Diritto d’autore, che riguarda le opere dell’ingegno creativo; i requisiti per la tutelabilità sono l’originalità e la novità; l’acquisizione avviene automaticamente con la creazione dell’opera e dura fino a 70 anni dalla morte dell’autore.
\end{itemize}

In generale i due tipi di diritti intellettuali difficilmente si sovrappongono o vengono confuse, eccezion fatta per
il software.

Fino a pochi anni fa, la situazione legislativa riguardante la proprietà intellettuale era la seguente:
\begin{itemize}
    \item Brevetti regolati dal Regio Decreto del 1939.
    \item Diritto d’autore regolato da una legge del 1941.
    \item Marchi regolati dal Regio Decreto del 1942.
\end{itemize}

Nel 2005 in materia di brevetti e marchi è sopravvenuta una riforma che ha accorpato tutto in un unico testo detto Codice della proprietà industriale (d.lgs. n.30/2005). Per il diritto d’autore invece è rimasta valida la stessa legge (Legge 633/1941 - LDA); quest’ultima ha tuttavia subito alcune anche importanti modifiche: 
\begin{itemize}
    \item Vengono aggiunti articoli riguardanti la tutela per i programmi per elaboratore.
    \item Viene armonizzata la durata della protezione del diritto d’autore.
    \item Viene disciplinata la tutela giuridica delle banche dati.
    \item Vengono aggiunti articoli per l’adeguamento alle direttive europee.
    \item Vengono definite delle disposizioni riguardanti la Società italiana degli autori ed editori.
\end{itemize}

Esiste una differenza sostanziale tra i concetti di diritto d’autore e copyright:
\begin{itemize}
    \item Copyright → Tipico dei paesi Common Law (anglo-americani); nato per promuovere l’industria culturale; letteralmente significa “diritto di copiare” ed è volto soprattutto a tutelare l’interesse del soggetto imprenditoriale che vuole investire sulla commercializzazione dell’opera (editore, produttore, ecc.).
    \item  Diritto d’autore → Più ampio rispetto al copyright, è tipico dei sistemi Civil Law; ci si concentra sull’autore, il quale anche dopo un’eventuale cessione dei diritti patrimoniali sull’opera può ancora conservare un certo controllo; l’opera creata dall’autore acquisisce un valore aggiunto, di tipo morale, rispetto al mero valore commerciale.
\end{itemize}