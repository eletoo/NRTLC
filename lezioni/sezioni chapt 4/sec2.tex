\section{Quadro di sintesi: origini ed attualità del DDA}

Il diritto d’autore è un’invenzione relativamente recente (prima c’era il mecenatismo e non si avvertiva la
necessità di tutelare i diritti su un’opera). L’esigenza della tutela si avverte con l’avvento della stampa
industriale; nasce quindi la compagine di soggetti classica del diritto d’autore:
\begin{itemize}
    \item Autore → Colui a cui spetta l’ideazione dell’opera.
    \item Editore → Colui che trasforma l’opera in bene di mercato.
    \item Distributore → Colui che mette a disposizione il bene.
    \item Fruitore → Colui a cui spetta l'acquisto e l’utilizzo degli esemplari dell’opera.
\end{itemize}

Il problema era che l’autore si trovava in una posizione svantaggiosa rispetto all’editore: da qui l’esigenza di riequilibrare la relazione autore-editore, con l’idea di favorire il diritto di sfruttamento esclusivo dell’opera da parte dell’autore.


Inizialmente si ha la tutela degli stampatori, solo più tardi si capisce che gli autori potrebbero non essere più incentivati a scrivere, poiché questo non consente più il loro stesso sostentamento; in Italia la prima legge organica sul diritto d'autore si ha nella seconda metà del 1800; a livello mondiale fu importante la Convenzione Universale sul Diritto d’Autore a Ginevra (1952), introdotta come alternativa alla precedente Convenzione di Berna per gli stati che non vi aderirono. Negli anni successivi venne istituita la WIPO, agenzia che redige ed amministra molti trattati internazionali.


Importantissimo è l’accordo TRIPS del 1994 che fissa gli standard per la tutela della proprietà intellettuale, al fine di colmare il divario nel modo in cui i diritti sulla proprietà intellettuale sono protetti in tutto il mondo; in particolare per il diritto d’autore assicura che i programmi per elaboratore saranno protetti come opere letterarie e descrive come dovrebbero essere protette le basi di dati.


A livello internazionale, si è verificata una sostanziale armonizzazione delle normative nazionali riguardanti il diritto d’autore in modo da garantire una tutela il più possibile omogenea; inoltre si è assistito ad un progressivo integrarsi tra copyright e diritto d’autore. Questo processo si è rivelato indispensabile soprattutto con la diffusione dell’informatica e delle comunicazioni elettroniche, che hanno ampliato le modalità di fruizione delle opere protette; la necessità di un approccio internazionale è stata evidenziata soprattutto dal carattere sovranazionale di Internet. Bisogna quindi rifuggire da un isolazionismo normativo che sfavorirebbe sia autori sia fruitori.


Il DDA nasce in risposta a fenomeni sociali ed economici: risulta infatti come una forma di incentivo alla creatività e di sviluppo culturale, caratterizzato in due modi differenti nei sistemi Common Law (prerogative all’editore) e Civil Law (prerogative all’autore). Linfa vitale del DDA è il meccanismo dell’esclusiva:
\begin{itemize}
    \item Il diritto esclusivo è la possibilità di escludere altri dall’esercizio di un diritto.
    \item Si tratta dell’unica possibilità per proteggere un’opera immateriale di tipo creativo da un uso indiscriminato.
    \item Potere contrattuale: è proprio il soggetto avente maggior potere contrattuale che detta le regole del gioco
\end{itemize}

Il quadro del diritto d’autore, in poco tempo, ha cominciato a subire cambiamenti a livello di base del sistema portando ad una crisi dell’impostazione classica. Infatti, non è più scontato l’uso di canali di realizzazione e distribuzione classici in quanto ora le opere possono essere convertite in un formato digitale e quindi apparire come file che può essere duplicato, inoltre si sono affermate nuove forme di creatività legate alla multimedialità e alla digitalizzazione.\bigskip


Compaiono inoltre nuovi players:
\begin{itemize}
    \item Service providers → Forniscono servizi che permettono di trasmettere, ricevere e organizzare materiale (es. email); spesso sono mere conduit (ovvero non vogliono responsabilità sui contenuti).
    \item Device and technology vendors → Implementano soluzioni di protezione (DRM, Digital Right Management) e valutano costi e responsabilità di queste.
    \item Content providers → Sono piattaforme contenutistiche, mettono quindi a disposizione del pubblico informazioni e opere di qualsiasi genere.
\end{itemize}

Tutti questi cambiamenti ed innovazioni (a livello tecnologico e anche normativo) sono impensabili al di fuori del terreno della convergenza. Esistono numerosi aspetti in conflitto tra loro: infatti il diritto alla remunerabilità dei contenuti (a favore dell’autore) è in contrasto con il diritto all’informazione e alla libera circolazione delle idee, alla libera manifestazione del pensiero e alla riservatezza (a favore del fruitore). Quindi la convergenza mette in crisi il sistema del DDA. 