\section{Funzionamento del Diritto d’Autore}

Secondo la LDA (legge del 1941) sono comprese nella protezione: opere letterarie, scientifiche (orali e scritte), musicali, teatrali, artistiche, cinematografiche, fotografiche, architetturali, programmi per elaboratore, disegni
industriali ecc. 

Le opere possono classificarsi in:
\begin{itemize}
    \item Opere collettive → Quelle costituite dalla riunione di opere o parti di opere.
    \item Elaborazioni creative → Rielaborazioni di opere già esistenti con un apporto creativo rilevante, come le traduzioni in altra lingua o le trasformazioni in altra forma artistica.
    \item Opere in comunione → Opere create con contributo di più persone.
\end{itemize}

Ma come si ottengono i diritti su una propria opera? A differenza del brevetto il DDA è automatico, cioè l’autore l’ottiene con la semplice creazione dell’opera.


Per quanto riguarda il deposito dell’opera invece, bisogna provare la paternità della stessa e i metodi per dimostrare la sua esistenza in una data certa sono vari (es. pubblicazione in un’edizione periodica, deposito presso enti pubblici, firma digitale ecc.); in ogni caso non c’è alcuna responsabilità per chi certifica la data di deposito, è l’autore che deve rispondere in caso di plagio.

E per quanto riguarda i requisiti per la tutela? Sono deboli, ma ci sono; conta il carattere creativo dell’opera, che si traduce in:
\begin{itemize}
    \item Originalità → Cioè deve esser frutto di un particolare lavoro intellettuale e deve essere legata all’impronta della personalità dell’autore.
    \item Novità → Si caratterizza in Novità Soggettiva, più legata all’autore e sovrapponibile alla originalità, e Novità Oggettiva, di maggiore importanza per le accertamenti di plagio.
\end{itemize}

Ciò che conta per il carattere creativo non è l’idea creativa in sé, ma la forma espressiva, cioè come tale idea è
stata estrinsecata dall’autore. Tale forma viene distinta in:
\begin{itemize}
    \item Forma esterna → Come appare l’opera nella sua versione originaria.
    \item Forma interna → Come viene esposta l’opera.
    \item Contenuto → L’argomento trattato, le informazioni e le idee.
\end{itemize}

La tutela riguarda solo la forma esterna e interna, senza estendersi al contenuto.


Il confine tra brevetto e DDA si vede quindi nel fatto che il DDA tutela la forma espressiva, mentre il brevetto tutela di più “l’idea in sé”.
Il diritto d’autore italiano viene classificato in:
\begin{itemize}
    \item Diritti patrimoniali→ Sono diritti esclusivi legati all’utilizzazione economica delle opere; si divide in:
    \begin{itemize}
        \item Diritti esclusivi di utilizzazione economica → Tali diritti comprendono la riproduzione, la trascrizione, l’esecuzione, la comunicazione al pubblico, la distribuzione, la traduzione, l’elaborazione, la pubblicazione, l’introduzione di modifiche e il noleggio dell’opera.
        \item Diritti connessi → Serie di diritti che nascono in capo a soggetti diversi dall’autore, ma la cui esistenza è direttamente “connessa” all’esercizio dei diritti d’autore (ad es. diritti sul merchandising).
        \item Equo compenso → Diritti di credito che sorgono in capo all’autore.
    \end{itemize}
    \item  Diritti morali → Sono attinenti alla sfera personale dell’autore e comprendono un riconoscimento di valore aggiunto (da cui ne consegue il diritto di rivendicazione della paternità e di opposizione a deformazioni o mutilazioni dell’opera); non è possibile cedere questo diritto e ha rilevanza giuridica e pratica.
\end{itemize}

Si pone poi il delicato problema di bilanciare gli interessi in gioco, ad esempio se in alcuni ambiti specifici le regole del DDA non sono opportune allora si sceglie di privilegiare l’interesse all’accesso ai contenuti creativi e dell’informazione; per fare ciò sono previste le libere utilizzazioni, cioè una sorta di zona franca in cui l’opera può essere usata liberamente; tuttavia prevale sempre il carattere eccezionale di tale istituto. In caso di violazione dei diritti d’autore, si può procedere con difese e sanzioni sia civili (azione inibitoria o risarcitoria) sia penali.