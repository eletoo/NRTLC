\section{La brevettabilità}
Sono brevettabili delle soluzioni nuove e inventive di problemi tecnici atte ad avere un'applicazione industriale.\bigskip

La legge (CPI - codice proprietà intellettuale, EPC - european patent code) stabilisce categorie e ambiti che non sono brevettabili. Non sono brevettabili:
\begin{itemize}
    \item Principi e metodi per attività intellettuale
    \item Scoperte e teorie scientifiche
    \item Programmi per elaboratore
    \item Presentazioni di informazioni
    \item Metodi per trattamento chirurgico o terapeutico
    \item Metodi diagnostici applicati al corpo umano o animale
    \item Razze animali e procedimenti essenzialmente biologici per ottenerle
    \item Invenzioni contrarie all'ordine pubblico o alla morale
\end{itemize}

Molte volte è l'inventore stesso che si danneggia da solo, in quanto potrebbe presentare pubblicamente una demo del prodotto prima ancora di aver depositato il brevetto. 
Il criterio per cui è brevettabile solo ciò che non è stato presentato pubblicamente è talmente stringente da tenere in considerazione anche le singole parti dell'invenzione: se anche una sola delle parti presentate pubblicamente non è stata brevettata allora essa non sarà più brevettabile. \bigskip

Le invenzioni sono brevettabili se sono \textit{nuove} e \textit{inventive}.

\subsection{Novità}
Un'invenzione può essere definita \textit{nuova} quando non è parte dello \textit{stato dell'arte}, ove per stato dell'arte si intende tutto ciò che era disponibile al pubblico prima della data in cui è stata depositata la domanda di brevetto.

\subsection{Inventiva}
Un’invenzione è considerata come implicante un’attività inventiva se, per una persona esperta del ramo, essa non risulta in modo evidente dallo stato della tecnica. \bigskip

Oggi la soglia di accesso alla tutela è relativamente bassa: non è richiesto un notevole progresso tecnico ma solo l'assenza di evidenza/ovvietà.

Spesso per pregiudizio si brevetta meno di quanto non si potrebbe perché non si percepisce il fatto di aver realizzato una invenzione brevettabile.La maggior parte delle invenzioni che sono oggi brevettate sono in realtà perfezionamenti di soluzioni già esistenti, che comunque garantiscono maggior competitività al prodotto e/o all’azienda. 

Inoltre l'invenzione deve essere descritta in modo sufficientemente chiaro e completo perché ogni persona esperta del ramo possa attuarla. 

Il brevetto è nullo se l'oggetto del brevetto si estende oltre il contenuto della domanda iniziale, dunque bisogna inserire tutte le informazioni nella domanda iniziale. 