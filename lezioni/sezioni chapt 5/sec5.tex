\section{Procedure di brevettazione}

\begin{itemize}
    \item Brevetti nazionali: Un brevetto nazionale è valido solo nel paese in cui è concesso; anche i non-residenti possono richiedere un brevetto. Un anno di priorità per successive domande di brevetto in altri paesi
    \item Brevetto Europeo: è equivalente a brevetti nazionali nei paesi in cui è convalidato (il richiedente sceglie i paesi)
    \item Patent Cooperation Treaty (PCT): un’unica domanda per oltre 140 paesi; dopo la fase iniziale, la domanda internazionale prosegue in un fascio di procedure d’esame nazionali. Le decisioni che implicano i costi maggiori possono essere posticipate fino a 30\textendash 31 mesi dal deposito
\end{itemize}

La procedura PCT permette di posticipare la decisione dei Paesi a cui estendere il brevetto e di non farlo subito dopo i dieci mesi dal deposito.