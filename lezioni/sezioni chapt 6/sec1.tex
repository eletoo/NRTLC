\subsection{Interoperabilità}

Si tratta della garanzia della capacità di componenti e prodotti tecnologici di cooperare e assicurare insieme un buon funzionamento di un sistema integrato senza sprechi di risorse (cioè senza dover utilizzare altri dispositivi/software che traducano le modalità di comunicazione da/a elementi che mancano di interoperabilità).

Abbiamo visto che l'interoperabilità è un componente fondamentale della concorrenza: aziende possono sfruttare l'assenza di interoperabilità dei loro prodotti per trarre vantaggio sul mercato (es. Apple, Microsoft\dots).

\begin{figure}
    \centering
    \includegraphics{interoperabilità.png}
    \caption{Livelli di interoperabilità}
    \label{Interoperabilita}
\end{figure}

A livello più basso, come mostrato in figura \ref{Interoperabilita}, l'interoperabilità si ha a livello tecnico e semantico; a più alto livello l'interoperabilità serve per arrivare a un'uniformità organizzativa e procedurale. 
Arriviamo poi al livello giuridico: l'interoperabilità permette di regolamentare in modo uniforme chi utilizza gli strumenti che ne godono. \bigskip

ATTENZIONE: non bisogna confondere l'\textit{inter}operabilità con l'\textit{intra}operabilità. 

\subsection{Neutralità tecnologica}
Possibilità di scegliere una tecnologia adatta ai propri bisogni senza che ci sia alcuna discriminazione a favore dell'impiego di un tipo particolare di tecnologia.

Portato a livello di quadro normativo che regola i settori di mercato tecnologico, questo principio non dovrebbe promuovere una determinata tecnologia a scapito di un'altra.
I meccanismi di concorrenza dovrebbero essere gli unici responsabili del vantaggio commerciale di una tecnologia rispetto a un'altra, ma tale tecnologia non deve essere esplicitamente favorita.

La neutralità tecnologica richiama il concetto di \textit{neutralità della rete}, ma i due concetti non sono da confondere:
la neutralità tecnologica è molto generale e va oltre i confini delle tecnologie (può essere relativa anche ad altri settori economici - es. trasporti), mentre la neutralità della rete è intrinsecamente legata alla rete.

Il principio di neutralità della rete garantisce che contenuti diversi in rete vengano trattati allo stesso modo (i.e. indipendentemente dal contenuto del payload i bit vengono trasmessi con la stessa qualità di servizio).
Questo principio chiaramente favorisce il principio di neutralità tecnologica. 

\subsection{Effetti di rete}
L'effetto di rete è un processo molto legato ai meccanismi di diffusione delle tecnologie. Si tratta di una forma di interdipendenza tecnologica, economica, giuridica, psicologica per la quale l'identità che il consumatore percepisce a seguito del consumo di un bene dipende (in modo positivo o negativo) dal numero di altri individui che consumano lo stesso bene. 

Ci possono essere fenomeni di rete che causano irrigidimento di mercato, ossia favoriscono certi prodotti rispetto ad altri perché tali prodotti hanno un effetto di rete positivo (es. WhatsApp molto utilizzato nonostante non sia l'applicazione migliore nel suo settore, ma l'uso da parte di tantissimi utenti la rende particolarmente diffusa). 

