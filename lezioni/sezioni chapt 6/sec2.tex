\section{Standard: definizioni ed esempi}

Il concetto di standard è centrale nelle tematiche di interoperabilità e neutralità tecnologica.

Per definizione (Treccani) lo standard è un modello o tipo di un determinato prodotto, o insieme di norme fissate allo scopo di ottenere l'unificazione delle caratteristiche (standardizzazione) del prodotto medesimo, da chiunque e comunque fabbricato. 
Anche, insieme degli elementi che individuano le caratteristiche di un determinato processo tecnico.


Il concetto di standard si estende dall'ambito tecnologico a tutto l'ambito manifatturiero e industriale. 
Esso agevola la progettazione e produzione, evitando un dispendio di risorse e garantendo maggiore accoglienza del prodotto sul mercato, ma anche gli utenti, che ricevono prodotti ideati sulla base di norme condivise e pertanto più probabilmente dotati di interoperabilità.

\subsection{Standard \textit{de jure}}
Lo standard si definisce \textit{de jure} quando è frutto di un regolare processo di analisi tecnica e definizione gestito da apposite organizzazioni.

Si tratta di uno standard definito legalmente come norma tecnica da parte di \textit{enti di formazione} o \textit{enti di standardizzazione}.

Secondo una Direttiva Europea del 1998 una \textit{norma} è la specifica tecnica approvata da un organismo riconosciuto a svolgere attività normativa.

\subsection{Standard \textit{de facto}}
Modelli di riferimento che solo per la loro elevata diffusione (effetti di rete più o meno spontanei/pilotati) vengono comunemente considerati standard, ma che in realtà non sono mai stati riconosciuti da apposite organizzazioni attraverso un regolare processo di standardizzazione. 
