\section{Dinamica storica della standardizzazione}

Possiamo osservare che il radicarsi di pratiche mirate alla formazione di standard industriali è direttamente proporzionale al grado di complessità e trasversalità di un determinato mercato, ma soprattutto al livello di convergenza tecnologica.

Possiamo individuare tre fasi evolutive della standardizzazione:
\begin{enumerate}
    \item Fase 1: in assenza di standardizzazione aziende diverse sono naturalmente portate a creare prodotti diversi 
    \item Fase 2: grazie alla liberalizzazione dei mercati si percepisce la necessità di standardizzazione internazionale per consentire l'inizio della convergenza tecnologica
    \item Fase 3: oggi la standardizzazione è un \textit{must} se si vuole massimizzare la convergenza tecnologica e rendere il proprio prodotto fruibile dal massimo numero di utenti possibile
\end{enumerate}


Secondo l'approccio statunitense l'attività di standardizzazione deve essere lasciata il più possibile al mercato, mentre in Europa si tende a voler coordinare maggiormente le direttive giuridiche con le norme tecniche.