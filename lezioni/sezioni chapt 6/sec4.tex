\section{Attività di normazione: principi, fasi, pubblicazione, uso}

L'iter che porta alla formalizzazione di uno standard è chiamato \textit{processo di standardizzazione}.

Il termine \textit{norma} in questo ambito non ha accezione giuridica ma di modello (standard \textit{de jure}).

L'ambiente di standardizzazione è un ambito in cui diversi portatori di interesse interagiscono: portano i propri interessi, competono al fine di farli valere e si accordano su uno standard. 
Un contesto in cui operano enti concorrenti è da un lato complesso e delicato e dall'altro molto stimolante, in quanto diversi portatori di interesse si scambiano idee di natura tecnica, giuridica, economica, \dots.

\subsection{Principi base}
\begin{itemize}
    \item Consensualità: si cerca sempre il massimo consenso (anche attraverso attività diplomatiche) tra chi è coinvolto nell'attività di normazione
    \item Democraticità: garantisce la rappresentatività dei portatori di interesse e permette loro di contribuire e partecipare alle decisioni
    \item Trasparenza: per prendere decisioni occorre essere informati e pertanto è necessario garantire l'accesso ai documenti del caso a tutti gli addetti ai lavori, in quanto nulla può essere deciso sulla base di cose non documentate
\end{itemize}

Alla fine del meeting vengono prodotti documenti di output basati su quelli di input e sulle scelte prese in modo democratico e consensuale. 
I documenti di output non sono necessariamente lo standard, ma possono essere relativi a nuove problematiche rilevate o indicazioni di lavoro in vista di un nuovo meeting. 

\subsection{Fasi}
Normalmente si lavora per commissioni tecniche che rappresentano le parti economiche e sociali interessate.

Poiché alcune attività di normazione sono molto complesse ed esistono al mondo diverse organizzazioni che si occupano di standardizzazione, è possibile che un ente prenda per buono lo standard di un altro ente. 
In questo caso si tratta di un lavoro di traduzione/rielaborazione più che di scrittura \textit{ex novo}.

\subsection{Pubblicazione e utilizzo degli standard}
Lo standard, normalmente, è un documento proveniente da un lavoro di valore e costoso, per cui è materiale coperto da diritto d'autore. Di conseguenza, l'uso di uno standard per un prodotto commerciale è soggetto a costi (è necessario acquistare lo standard per poterlo usare). 

Le altre entrate degli enti di standardizzazione provengono dalle quote di partecipazione dei comitati tecnici che vogliono prendere parte alle procedure di standardizzazione.

Oltre alle tutele giuridiche per l'accesso alla documentazione, possono sussistere dei diritti di proprietà industriale (brevetti) sulle soluzioni tecniche contenute e descritte nello standard. 
Per il principio di trasparenza si sono sviluppate politiche all'interno degli enti di standardizzazione che impongono che all'inizio dei lavori si dichiari l'esistenza di eventuali brevetti sulle tecnologie che verranno proposte per l'inserimento nello standard. 

Ciò avviene per evitare che una volta definito lo standard salti fuori che una tecnologia usata nella sua definizione è coperta da brevetto, implicando costi inaspettati relativi alla proprietà industriale.
