\section{Problematiche in materia di standardizzazione}

Avere uno standard è sempre un beneficio per lo sviluppo della tecnologia ad esso legata? I problemi principali sono legati all'innovazione tecnologica, alla proprietà intellettuale e alla concorrenza. 

\subsection{Standard e innovazione tecnologica}
Nel momento in cui fissiamo uno standard stiamo cercando di cristallizzare un modello di riferimento a cui per un certo
periodo dovrà conformarsi lo sviluppo di quella specifica tecnologia, ma la tecnologia è in continua evoluzione (cosa che vanifica le standardizzazioni).

Si rischia inoltre che una standardizzazione mal strutturata porti a situazioni di stallo e irrigidimento del mercato, per cui il superamento di uno standard obsoleto costerebbe di più di quanto non si guadagnerebbe aggiornandolo. 

Si veda a tal proposito lo standard di compressione audio-video utilizzato ancora oggi nonostante sia vecchio di 20 anni: si tratta di uno standard molto più arretrato rispetto a quanto la tecnologia non consentirebbe di fare, ma la standardizzazione di un nuovo algoritmo più efficiente costerebbe troppo in termini di tempo e costi di aggiornamento. 

\subsection{Standard e gestione della proprietà intellettuale}
Chi propone l'inclusione di tecnologie negli standard deve essere trasparente e dichiarare se esse sono protette da qualche forma di brevetto. 
Gli enti di normazione prevedono policy di trasparenza da questo punto di vista.

\subsection{Standard, concorrenza e problematiche antitrust}
C'è particolare attenzione nel sorvegliare le aziende che stipulano accordi per la ripartizione del mercato o per la standardizzazione di determinati comportamenti/tecnologie (cartelli). 

A livello di normativa bisogna impedire che le aziende si accordino (es. per fissare il prezzo di un prodotto) \textit{tranne} nel caso in cui tali accordi migliorino la produzione o la distribuzione dei prodotti. \bigskip

Problematiche relative al \textit{patent pooling}.

\subsection{Norme Armonizzate}
Specificità europea: consente la sinergia tra norme e leggi. Le norme armonizzate sono un riferimento per la produzione e progettazione di beni/servizi sicuri e ambientalmente compatibili (es. applicazione del marchio CE soggetta a normative stringenti in termini di sicurezza per la salute dei consumatori).

\subsection{Open Standard}
Uno standard è aperto quando:
\begin{itemize}
    \item è adottato e mantenuto da un'organizzazione non-profit e il cui sviluppo avviene sulle basi di un processo decisionale aperto 
    \item il documento di specifiche è disponibile liberamente oppure a un costo nominale
    \item eventuali diritti di copyright, brevetti o marchi registrati sono irrevocabilmente concessi sotto forma di royalty-free
    \item non è presente alcun vincolo al riuso, modifica o estensione dello standard
\end{itemize}

\subsection{Classificazione degli standard}
\begin{itemize}
    \item Livello 0: chiuso o proprietario
    \item Livello 1: divulgato
    \item Livello 2: concertato (specifiche ottenute tramite consultazione e collaborazione)
    \item Livello 3: concertato aperto (come livello 2 ma sotto la guida di un ente \textit{super partes})
    \item Livello 4: aperto de jure (stabilito da organismi internazionali)
\end{itemize}
