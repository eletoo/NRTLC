\chapter{Privacy - Avv. Vincenzi}

La legge sulla privacy in Italia è datata 1995: dalla metà degli anni '90 si inizia a percepire la necessità di introdurre regolamentazione in campo delle tecnologie. Sulla base della direttiva europea del '95 fu varata la norma italiana del '96. 
Solo nel 2016 arriva il regolamento europeo (GDPR), che entra in vigore solo a partire dal 2018 (ancora oggi non in tutti gli Stati).
Il GDPR ha un valore maggiore rispetto alle leggi statali, in quanto si tratta di una legge europea.

Il termine più corretto per riferirsi alla privacy dovrebbe essere protezione dei dati personali. Si tratta in realtà di due concetti definibili in modo distinto: in America prevale la  ``teoria del recinto'', in Europa la teoria ``habeas data''. La teoria del recinto prevede che una volta che si entra nel recinto, si può fare quello che si vuole dell'erba al suo interno (es. una volta accettato di utilizzare i servizi di un social network il social è autorizzato a utilizzare i dati dell'utente come vuole); il principio dell'habeas data invece garantisce che i dati di una persona siano comunque di sua proprietà e che la persona possa scegliere in modo arbitrario come essi vengono utilizzati (analogamente a cosa succede quando si deposita del denaro in banca e ci si riserva la possibilità di ritirarlo, modificare gli investimenti, etc.).
Il primo articolo del GDPR assicura che la legge protegge gli individui come persone attraverso la protezione dei dati personali. Una qualsiasi legge è in effetti uno strumento per proteggere le persone fisiche.

Possiamo distinguere tra:
\begin{itemize}
    \item Privacy by design: qualsiasi sistema (fisico o informatico) deve rispettare già dall'impostazione della struttura della privacy fin dall'inizio. Gli spazi informatici o fisici devono rispettare fin dall'inizio la privacy (es. le farmacie sono strutturate per privacy by design separando i clienti che vengono serviti da quelli in attesa tramite una riga sul pavimento, in modo da garantire la privacy)
    \item Privacy by default: bisogna trattare il numero minimo di dati necessari (es. per un servizio di e-commerce potrebbe essere sufficiente raccogliere come dati degli utenti nome, cognome, indirizzo e numero di telefono)

\end{itemize}

Tutte le norme hanno tre ambiti di applicazione:
\begin{itemize}
    \item Materiale: il presente regolamento si applica al trattamento interamente o parzialmente automatizzato di dati personali e al trattamento non automatizzato di dati personali contenuti in un archivio o destinati a figurarvi. Si definisce archivio qualsiasi insieme strutturato di dati personali accessibili secondo criteri determinati. L'unica eccezione in cui non di applica il GDPR è il caso in cui il trattamento di dati avviene da una persona fisica per l'esercizio di attività a carattere esclusivamente personale o domestico. 
    \item Temporale
    \item Spaziale o territoriale: non ci sono problemi quando il titolare è stabilito in Europa (sede legale in Europa); poiché i social media non hanno sede legale in UE, l'Unione ha reso il GDPR applicabile anche in questo caso specificando che i dati protetti sono quelli degli utenti che si trovano nell'Unione, indipendentemente dal Paese di residenza/provenienza dell'utente
\end{itemize}
La legge chiaramente collide con l'assenza di concezione temporale che caratterizza il web. 

Definiamo dato personale qualsiasi informazione riguardante una persona fisica identificata o identificabile; si considera identificabile la persona fisica che può essere identificata direttamente o indirettamente, con particolare riferimento a un identificativo come il nome, un numero di identificazione, dati relativi all'ubicazione, un identificativo online o a uno o più elementi caratteristici della sua identità fisica, fisiologica, genetica, psichica, economica, culturale o sociale.
N.B. per ``persona fisica'' si intende un individuo in vita (non una società, associazione, etc.).

I dati personali possono essere distinti in:
\begin{itemize}
    \item Dati comuni
    \item Dati identificativi
    \item Dati espliciti
    \item Dati impliciti secondari o derivati o inferti: sono di fatto i dati più importanti, soprattutto dal punto di vista dell'e-commerce
\end{itemize}

I dati particolari o sensibili sono dati personali che rivelino l'origine razziale/etnica, le opinioni politiche, le convinzioni religiose o filosofiche, l'appartenenza sindacale, dati genetici, biometrici, relativi alla salute, alla vita sessuale o all'orientamento sessuale di una persona. In sostanza si tratta di dati che potrebbero causare, se utilizzati per scopi malevoli, discriminazioni nei confronti di un individuo. 
Tra i dati particolari si trovano i dati biometrici, ossia dati ottenuti da un trattamento tecnico specifico dei dati relativi alle caratteristiche fisiche, fisiologiche o comportamentali di una persona fisica che ne consentono o confermano l'identificazione univoca, quali l'immagine facciale o i dati dattiloscopici. 

Cosa si intende per trattamento? Qualsiasi operazione o insieme di operazioni compiute con o senza processi automatizzati applicati a dati personali o insiemi di dati personali come la raccolta, la registrazione, l'organizzazione, la strutturazione, la conservazione, l'adattamento o la modifica, estrazione, consultazione, uso, comunicazione mediante trasmissione, diffusione o qualsiasi altra forma di messa a disposizione, fino alla cancellazione o distruzione.

L'interessato è la persona fisica a cui appartengono i dati; il titolare del trattamento è colui che determina le finalità e i mezzi del trattamento dei dati personali (non necessariamente il legale rappresentante, può essere persona fisica o giuridica); il responsabile è colui che tratta i dati per conto del titolare; gli autorizzati sono quanti hanno accesso ai dati ma non possono trattarli se non sotto istruzione da parte del titolare del trattamento.

\section{I principi del trattamento}
I dati devono essere:
\begin{itemize}
    \item Trattati in modo lecito, corretto e trasparente. Per trasparenza si intende che il cliente deve essere al corrente di tutte le modalità di trattamento dei dati. 
    \item Raccolti per finalità determinate, esplicite e legittime e successivamente trattati in modo che non sia incompatibile con tali finalità. Le finalità sono la parte più complessa da stimare, in quanto molte di esse non sempre sono note in modo esplicito
    \item Adeguati (sufficienti), pertinenti e limitati (minimi necessari) rispetto alle finalità per cui sono trattati. Si tratta di un'applicazione del principio di privacy by default (minimizzazione dei dati, i.e. principio del minimo privilegio)
    \item Esatti e se necessario aggiornati: è necessario rettificare tempestivamente eventuali inesattezze nei dati
    \item Trattati in maniera da garantire un'adeguata sicurezza mediante misure tecniche e organizzative adeguate. Le misure tecniche agiscono tramite mezzi concreti sugli oggetti da proteggere (password)
\end{itemize}

Il trattamento dei dati è lecito se e solo se:
\begin{itemize}
    \item L'interessato ha espresso il consenso al trattamento dei propri dati per una o più specifiche finalità
    \item Il trattamento è necessario per l'esecuzione del contratto di cui l'interessato è parte; in genere se il dato personale viene meno e il contratto non è più rispettato allora significa che quel dato è necessario per l'esecuzione del contratto
    \item Il trattamento è necessario per adempiere a obblighi di legge
\end{itemize}

Il consenso dei dati viene utilizzato solo se non si è coperti da contratto o obblighi di legge; il consenso è una manifestazione libera, specifica, informata e inequivocabile dell'interessato con la quale egli mostra il proprio assenso al fatto che i dati siano oggetto di trattamento. 

Il titolare del trattamento deve mettere in atto misure e tecniche organizzative adeguate per garantire ed essere in grado di dimostrare che il trattamento è effettuato conformemente al GDPR. 